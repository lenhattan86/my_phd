\section{Evaluation}
\label{sec:evaluation}

We evaluate \name through both experiments on real systems and numerical simulations. Results highlight:
\begin{itemize}
	\item \name reduces the average job completion time by a large fraction compared to existing methods and other baselines. \name retains significant improvements across various settings, even with relatively large estimation errors and overheads. 
	\item \name achieves comparable performance of \SRPT under a wide range of settings. Recall that \SRPT allows preemption at no cost and optimizes performance only with no fairness restriction. Informally, it can be considered as an unrealistic lower bound for the average job completion time.  
    \item \name provides fairness among users that is better than or comparable to existing fair allocation, and does not result in starvation for users with long jobs. Actually, \name provides the best progress for these users. 
	%\item \name retains significant improvements across various settings, even with relatively large estimation errors and overheads\todo{we need more figures to illustrate this point}.
\end{itemize}

\subsection{Setup}

\paragraph{Cluster} We setup Kubernetes with GPU support on a cluster of one master node, 8 CPU workers, and 4 GPU workers.
The master node does not execute any jobs. Instead, it coordinates the worker nodes and runs the job estimation tools. 
Each CPU worker is a \textit{xl170r} server from Cloudlab~\cite{cloudlab} with 20 virtual CPU cores and 64GB RAM.
Each GPU node is a \text{p2.xlarge} instance from Amazon EC2 with 1 K80 GPU, 4 CPU cores and 61GB RAM.
There are 4 users, which is increased in the simulations.

\paragraph{Workload} There are 4 users sharing the cluster. Each user has 10 popular Tensorflow jobs, e.g., Googlenet, Lenet, and Alexnet. The job configurations such as batch sizes and batch numbers are different, resulting in the speedup rates of using one GPU versus one CPU ranging from 1.8 to 10. 
For jobs on CPU, the number of threads is set at 19 to best utilize the virtual cores while leaving one core for other services on each node. 
We run a small sampling job for each real job to obtain the parameters for both CPU and GPU configurations, as we discussed in Section~\ref{sec:profiling}. 
The total overhead of sampling jobs is 3\% of the real jobs. We vary the settings in Section~\ref{sec:sensitivity} to evaluate the impacts. 
%Note that we cannot use 20 cores because there are other services are running on each node.
%\new{The fairness level is set at 0.5, meaning, we pick 2/4 of all users with the least fairness scores for scheduling overtime. }

\paragraph{Simulator} While experiments on real systems provide most realistic evaluations, it is costly to run in large scale over a long period. To fully evaluate \name, we implement a Java-based cluster simulator, which emulates the cluster with multiple resources, e.g., CPU, GPU, and memory. We validate the accuracy of the simulator by comparing its results to those from the real experiments over the cluster (Figure~\ref{fig:avgCmplt_exp}). 
%Jobs on the simulator can interchange between CPUs and GPUs.
%\todo{Add the setup information}
% For fair-cost comparison, we consider the CPU cores to GPU rate is 32.
There are 20 GPUs, 20 CPUs with 20 cores each, and 1280 GB RAM. %\todo{Tan, please check}.
%There are 10 users.

For numerical simulations, we use the workload trace from the Google cluster \cite{google-traces} to generate arrival times for Tensorflow jobs.
%The arrival times of our workload are the arrival times in the Google trace.
There are 10 users and over 1000 jobs for each user. %With more users, the improvements of \name is even larger. \todo{can we run a simulation with 20 users to validate this?}
By default, the fairness level $\alpha$ is set at $0.1$, meaning, we schedule jobs from the $1/10$ of all users who have the least progress whenever a node becomes available. The estimation errors are based on our experiments with sampling jobs whose sizes are 3\% of the corresponding real jobs and are around 10\% as we discussed in \S\ref{sec:profiling}. The impacts of the fairness level, estimation errors, and overheads are studied in \S\ref{sec:tradeoffs}, \S\ref{sec:estimation_err}, and \S\ref{sec:overhead}, respectively.
% \todo{Tan, please update the reference}.
% \new{
% The standard deviation of estimation errors is set 10\%.
% The total overhead of sampling jobs is 3\% of the real job.
% The fairness level, estimation errors, and overheads and  are studied and varied in \S\ref{sec:tradeoffs}, \S\ref{sec:estimation_err}, and \S\ref{sec:overhead}.
% }

\paragraph{Metrics} %We consider average job completion time, and fairness.
To evaluate the performance, we measure the average completion time of all jobs under \name and baseline algorithms.
%Users expect to have shorter average completion time.
We use standard deviation of progresses across users to evaluate fairness. %\todo{This is to be updated.}
For starvation, we focus on the progress of users with longer jobs. %we compare the top 1\% job waiting time.

\subsection{Baselines} 
\label{sec:baselines}
We compare \name to the following methods.

\emph{\ESRP} (equal share with shortest job first): \ESRP divides all resources equally among users statically. 
For a particular user, whenever a resource becomes available, \ESRP picks the job with the shortest processing time on this resource. For instance, if all jobs prefer GPUs, \ESRP first fills up all available GPUs with shortest jobs based on their processing time on GPU, and then fills available CPUs with the shortest jobs using CPU configurations. \ESRP needs the estimator to predict the processing time in different configurations.
%If a user does not use up her allocated resources, it is equally shared among users with queued jobs. 
% \new{
% To pick the job configuration, \ESRP first fills up all available GPUs configurations of waiting jobs and fills up available GPUs with the shortest ones.
% When GPU is full and there are waiting jobs, \ESRP fills available CPUs with the shortest ones using CPU configurations.
% If a user does not use up his allocated resources, it is equally shared by the other users.
% }
%\del{We first use SJF to schedule all jobs to GPU then use SJF to schedule remaining jobs to CPU.} %\zhenhua{This does not make sense. If you already put all jobs to GPU, then there is no jobs left.}
%\tanle{updated.}

\emph{\DRFFIFO} (online DRF with FCFS): %\del{If a job's completion time on CPU is shorter than that on GPU, it is placed on CPU, and vice versa. }
%\todo{Xiao, how is DRF calculated? Does it need profiling to know the demand, or you simply run FIFO and pick the user with the lowest progress?}
Whenever a resource becomes available, \DRFFIFO schedules the first job of the user with the slowest progress.  
Therefore, jobs are processed in a First-Come-First-Served manner within every user. 
For job configuration, we assume users have some preference. If all jobs prefer GPUs, \DRFFIFO always picks the GPU configuration. 
\DRFFIFO does not need the estimator to pick the configuration, and users' progress can be updated whenever a job is completed. 
%     Each job only has a single configuration that is chosen by users.
%     As DRFFIFO does not have 
% 	Hence, a job only has a single configuration and DRF does not need to handle the interchangeability.
%\tanle{updated.}



% Since \DRFFIFO schedules jobs based on their arrivals, it does not need the estimator.
% Like existing systems like Kubernetes \cite{kubernetes}, every job is predefined with a single configuration.
% In practice, users prefer to run their jobs on GPUs because GPUs are often more powerful than CPUs.
% So, we assume all jobs are configured to run on only GPUs with a large memory request (maximum memory of a node).
% Online DRF is used to allocate the single-configuration jobs with 2 resources (GPU and memory) like \cite{drf}.
%\zhenhua{I think we can add another baseline of just FIFO} 

	
\emph{\DRFSJF} (DRF with shortest job first): \DRFSJF is similar to \DRFFIFO, but within each user, jobs are scheduled in a shortest-job-first manner. Therefore, the estimator is needed. Each user relies on the estimation to pick whether CPU or GPU for each job configuration. 


% When a GPU (or CPU) becomes available, \DRFSJF picks the user with the lowest progress and schedule her job with the smallest processing time on GPU (or CPU, respectively).
% We consider another baseline using DRF that uses SJF to schedule the jobs. \todo{Xiao, again, how is DRF calculated?} \tanle{updated.}
% \new{
% Since \DRFSJF schedules jobs within a user, it need the estimator to estimate the completion times of jobs on both CPU and GPU configurations.
% \DRFSJF chooses the most effective configuration based on the estimated completion times.
% Online DRF is used to allocate the single-configuration jobs with 2 resources (GPU and memory) like \cite{drf}.
% }
% \zhenhua{From the results, CPU utilization of \DRFSJF is low. Does this mean when a CPU is available, \DRFSJF does not always schedule jobs on that? If so, we need to adjust the description accordingly.}

\emph{\DRFExt}: \DRFExt uses some average speedup rate to convert GPU resources to the corresponding CPU ones, e.g., if the speedup rate is 10, 1 GPU is considered 10 CPU. %\todo{We don't need transfer rate. Use the number of CPU cores to update Figure 2 to compare between one CPU as a whole and one GPU card.}.
%\tanle{don't get it.}
Then the problem is simplified to the original multi-resource allocation without interchangeable resources, and online DRF is applied. 
Within each user, jobs are processed in a shortest-job-first manner, and therefore the estimator is needed.
%\zhenhua{The following sentences are not clear.}
% We assume that the we can convert 3-resources to 2-resource cluster based on the average transfer rate of all jobs. If transfer rate is 10, 1 GPU can be exchanged with 10 CPU cores.
% We apply DRF to 2-resource cluster.
% As a job has 2 configurations, DRFExt needs to choose which configuration to run.
%\tanle{updated.}
% \new{
% \DRFExt picks the configuration of a job like \ESRP does.
% When GPU is available, \DRFExt assigns the resources to the jobs with shortest completion times on GPU first.
% When GPU is full, \DRFExt assigns the resources to the jobs with shortest completion times on CPU.
% }

\emph{\SRPT}: At any time, the job with the shortest \emph{remaining} processing time is executed, which requires preemption. This approach is \emph{unrealistic} in many real systems such as Kubernetes because jobs cannot be paused and moved from one resource to another, or even a different resource, without large overheads.
However, \SRPT is good at minimizing the average job completion time, and therefore serves as a goal for \name to achieve. Note that \name used in this section does not use preemption for conservative evaluations of the improvements. 



%We use makespan for evaluating efficiency.
%Given the same number of jobs, the makepan is the time for a scheduler to allocate resources and complete all jobs.
%Hence, the shorter makespan shows the better efficiency.
%Equal Sharing (ES) the resources are fair in terms of the amount of received resources.
%We use ES as the baseline for fairness.
%If performance of a user is worse than ES, it is not fair for him to use \name.


\subsection{Results}

%\paragraph{Detailed resource allocation.}
%Figure \ref{fig:res_usage} shows the resource usage and fairness scores of 4 users overtime.
%As sample (sampl.) jobs for all users are highly prioritized as it is necessary to estimate the job completion times.
%It takes 6 minutes in total to finish all the sample jobs.
%Athough all users prefer GPU resources, not all of them are allocated to use GPUs.
%User 1 and User 3 take turns to share CPU as they have lower speed up rates on CPUs.
%user 1's jobs are scheduled first because his jobs on CPU are actually shorter than user 3's.
%The jobs of User 2 and User 4 have higher speed-up rates share GPUs and finish quickly.
%Not the resource usage but the fairscore actually decides the turn of users for resource allocation.
%We set $\alpha=0.5$ here so 2 users are chosen each turn.
%
%\begin{figure}[H]
%	\centering
%	\subfloat[\name]{\includegraphics[width=1.0\linewidth]{figs/res_usage_allox}}
%	\caption{[Cluster] 4 users share the resource and we update fairness scores overtime to maintain the fair allocation. Sampling (sampl.) jobs are prioritized and scheduled first. \todo{fair score \& resource usage are not well synchronized as they were logged using different ways. Need to fix the plot for fairness I just saved the fairness score whenever it changes.}}
%	\label{fig:res_usage}
%\end{figure}

We first evaluate \name through real experiments and validate the accuracy of the simulator in Section~\ref{sec:eval-cluster}. Results from the simulator are discussed in Section~\ref{sec:large-sim}.
% First, we shows the improvement in terms of average completion time in \S\ref{sec:large-sim}.
% Then, we present the trade-offs between performance and fairness of \name in \S\ref{sec:tradeoffs}. 


\subsubsection{Experiments over a cluster}
\label{sec:eval-cluster}

%\paragraph{Average completion time.}
Figure \ref{fig:avgCmplt_exp} illustrates the average job completion time
under \name and other baselines through experiments and the simulator we developed. 
%\todo{Xiao \& Tan, please check the following paragraph to see if it accurately describes the insights. We need more details.}
First, Allox reduces the average completion time significantly compared to other baselines. In particular, \DRFFIFO and \DRFSJF do not fully utilize the CPU resources as shown in Figure~\ref{fig:res_util}, therefore incur longer waiting time. 
\ESRP and \DRFExt reduce the waiting time by increasing the CPU utilization (Figure \ref{fig:res_util}).
%\new{
Although \name has similar CPU and GPU utilization compared to \DRFExt and \ESRP, \name outperforms \DRFExt and \ESRP by better job scheduling and configuration selection. This is highlighted by the significant reduction in job processing time. 
%the placement of \name are more efficient than \DRFExt and \ESRP. 	
%}


%\zhenhua{Can we show the CPU/GPU utilization?} \tanle{added.}



% Although \DRFFIFO it does not suffer from the estimation overheads, the jobs are queued up on GPU and results in the worst performance.
% \DRFSJF performs much better than DRF as jobs are scheduled in the shortest job first manner.
% \ESRP is far from the optimal solution.
% Although we computed the average resource transfer rate for \DRFExt by profiling all the jobs, it does not perform well.

Figure \ref{fig:avgCmplt_exp} validates that the average completion times in the simulation and experiments are consistently similar.
This allows us to perform larger-scale evaluations using the simulator.% more extensive results without running time-consuming experiments.

\begin{figure}[t]
	\centering
%{\includegraphics[width=0.7\linewidth]{figs/avgCmplt_exp}}
{\includegraphics[width=1.0\linewidth]{figs/avg_comparison_group}}
	\caption{[Cluster] \name reduces the average job completion time. For each algorithm, the first bar shows results from experiments, and the second bar is from our simulator.} %Specifically, Allox reduces the completion time by 36\% compared to \DRFFIFO, 24\% compared to \DRFSJF, 25\% compared to \ESRP, and 27\% compared to \DRFExt. For each algorithm, the first bar shows results from experiments, and the second bar is from our simulator.} %Wrong estimation in some jobs in DRFS makes running time are longer than DRFF.}
	\label{fig:avgCmplt_exp}
\end{figure}

%\begin{figure*}[h]
%	\centering
%	\subfloat[\DRFFIFO] {\includegraphics[width=0.15\linewidth]{figs/res_util_DRFF} \label{fig:res_util_DRFF}} 
%	\hspace{0.1cm}
%	\subfloat[\DRFSJF] {\includegraphics[width=0.15\linewidth]{figs/res_util_DRFS} \label{fig:res_util_DRFS}}  
%	\hspace{0.1cm}
%	\subfloat[\ESRP] {\includegraphics[width=0.15\linewidth]{figs/res_util_ES} \label{fig:res_util_ES}} 
%	\hspace{0.1cm} 
%	\subfloat[\DRFExt] {\includegraphics[width=0.15\linewidth]{figs/res_util_DRFExt} \label{fig:res_util_DRFExt}}  
%	\hspace{0.1cm}
%	\subfloat[\name] {\includegraphics[width=0.15\linewidth]{figs/res_util_AlloX} \label{fig:res_util_AlloX}}     
%	\caption{[Cluster] Resource Utilization.}
%	\label{fig:res_util}
%\end{figure*}

\begin{figure}[b]
	\centering
	 \includegraphics[width=0.7\linewidth]{figs/res_util}  
	\caption{[Cluster] CPU and CPU utilization.}
	\label{fig:res_util}
\end{figure}

%\begin{figure}[H]
%	\centering
%{\includegraphics[width=0.7\linewidth]{figs/avg_comparison}}
%	\caption{[Cluster] vs [Simulator]}
%	\label{fig:avgCmplt_exp}
%\end{figure}


%\subsection{Simulation results}


%\emph{Dynamic demand.} Whey jobs are heterogeneous, demand is dynamic. We use simulator to evaluate the performance for demand detection and reallocation.
%\todo{present the dynamic demand case: what traces we use or what random distribution we are based on.}

\subsubsection{Simulation results}
\label{sec:large-sim}

%\todo{introduce other parameters in evaluation if any.}

%\paragraph{Average completion time.}
Figure \ref{fig:sim_large_scale} shows our simulation results. With a larger scale and more jobs, \name ($\alpha = 0.1$) consistently outperforms \DRFFIFO, \DRFSJF, \ESRP, and \DRFExt even more, reducing the completion time by 95\%, 84\%, 88\%, and 53\%, respectively.
Impressively, \name is not far (30\%) from \SRPT that only minimizes the average completion time without considering fairness, and with preemption allowed. 
When we focus on the longest 1\% jobs, \name has even larger improvements and beats \SRPT. This is not surprising because \SRPT prioritizes shorter jobs. 

%\begin{figure}[H]
%	\centering
%	{\includegraphics[width=1\linewidth]{figs/avgCmplt_m1}}    
%	\caption{[Simulation] AlloX outperforms others and is not far from the impractical SRPT in large-scale simulations. \todo{update the text accordingly.}}
%	\label{fig:sim_large_scale}
%\end{figure}

\begin{figure}[h]
	\centering
	\subfloat[all jobs]{\includegraphics[width=0.47\linewidth]{figs/avgCmplt}}
	\subfloat[1\% longest jobs]{\includegraphics[width=0.47\linewidth]{figs/avgCmplt_99}}    
	\caption{[Simulation] AlloX outperforms others and is not far from the impractical SRPT in large-scale simulations.}
	\label{fig:sim_large_scale}
\end{figure}



%\paragraph{Performance overtime} 
%Scheduling jobs to minimize the average completion time may cause large jobs suffering from starvation.
%The error bars in Figure \ref{fig:sim_large_scale} show the average completion times of the top 5\% and 1\% of jobs. 
%Since \DRFSJF, \ESRP, \DRFExt, and \SRPT based on shortest job first schedules the long jobs last, the long-jobs can be starved.
%When there are many queued up jobs like \DRFSJF, the starvation is much more significant. \zhenhua{Why many queued up jobs in \DRFSJF?}
%\new{ \DRFSJF has much longer waiting time than \ESRP because it cannot handle the incoming load that results in more queued up jobs over time.
%}
%\todo{The starvation in SRPT is not noticeable.}


%To provide more details regarding the comparison , we show the job arrivals and average JCT over the time horizon in Figure \ref{fig:perf_overtime}.
%If the average completion times are very high at some periods, there are unscheduled jobs hungry for resources. \zhenhua{Why? Can that be from larger job sizes during those periods or longer queues instead of starvation?}
%\zhenhua{The following description is not clear. Do spikes imply starvation?}
%Recall that \DRFFIFO has the worst performance, it does not have large spikes.
%It is because \DRFFIFO serves the jobs in the FIFO manner so the jobs do not suffer from starvation.
%\DRFSJF has significant spikes.
%Basically, it scarifies starvation for performance. 
%Similarly, \ESRP, \DRFExt, and \SRPT also have noticeable spikes due to starvation.
%In contrast, \name 's average completion time of is the smoothest over time.
%Meaning, \name suffers least from starvation.


\begin{figure}[h]
	\centering
	\hspace{-0.3in}{\includegraphics[width=1.1\linewidth]{figs/perf_overtime}}
	\vspace{-1in}
	\caption{[Simulation] \name consistently maintains the best performance over time, especially during the period with high arrival rates. \name also provides consistently small disparity across users' progress.}
	\label{fig:perf_overtime}
\end{figure}

To provide more details regarding the comparison, we show the job arrivals, average job completion time, and the standard deviation of progresses across users over time in Figure \ref{fig:perf_overtime}.
The average completion times of \name and \SRPT are consistently better over time compared to other baselines. Note the completion time is shown on a logarithmic scale.
When the arrival rates are high, the completion time of other baselines are much higher than that of \name. Not surprisingly, \DRFFIFO cannot process the jobs fast enough so queues build up. 
This highlights by effective scheduling and configuration selection, \name processes jobs faster and therefore allows for high arrival rates. %The large spike around the end is because some jobs running on CPUs finally finish, and artificially increases the average job completion time during this period. 

The figure also shows the disparity across users' progress over time. In this case, \SRPT and \DRFExt are much worse than \name, while DRFF, DRFS, and ES are a little better than \name (all users progress at similar, but much slower rates compared to \name). 
While the disparity in \SRPT is intuitive, \DRFExt fails to provide fairness because it ignores the different speedups of users, and instead use some averaged value to allocate resources. 

Overall, \name provides near optimal performance (similar to \SRPT) and maintains good fairness over time. 

\paragraph{Starvation}

In the extreme case, some users starve under schedulers like \SRPT. 
Figure \ref{fig:job_completed} shows the number of completed jobs of the user with longer jobs than others.
As expected, \SRPT performs the worst with most jobs unfinished.
In contrast, \name provides the best progress of this user because it maintains fairness similar to other baselines and is more effective. 


\begin{figure}[h]
	\centering
	\includegraphics[width=0.7\linewidth]{figs/job_completed}  
	\caption{[Simulation] \name provides the best progress for users with longer jobs, and prevents starvation.}
	\label{fig:job_completed}
\end{figure}

\paragraph{Performance and fairness trade-offs}
\label{sec:tradeoffs}

Figure \ref{fig:fairness_alpha} shows the trade-offs between performance (average completion time) and fairness. 
We vary the parameter $\alpha$ from 0.1 to 1 (smaller means fairer) and compare the performance with \ESRP and \SRPT.
%Small $\alpha$ is better in terms of fairness.
This shows with larger $\alpha$, \name approaches \SRPT in performance. The small gap (9\%) between \name and \SRPT when $\alpha=1$ is due to the fact \SRPT is preemptive at no cost while \name does not allow preemption. 
However, the unfairness in terms of standard deviation of users' progress also increases with larger $\alpha$. Normally, a small $\alpha$ around 0.2 is good at providing large performance improvements at little cost of fairness. 
In practice, a system operator can pick the tradeoff according to her situation. 



%When $\alpha$ is large, \name has more queued jobs for scheduling so it performs better.
%Interestingly, it can reach closely to the performance of unrealistic \SRPT (2.2\%).


%We use Jain's fairness index (JFI) to measure the fairness.
%For fair comparison, we use resource usage as input for JFI for \DRFFIFO, \DRFSJF, \ESRP, and \DRFExt.
%Meanwhile, we use fairness cores as the input for \name and \SRPT.
%\SRPT is the best in terms of average completion time but worst in fairness.
%When we change $\alpha$, \name closes the performance gap with \SRPT.
%Meaning, \name can sacrifice fairness to gain performance.


\begin{figure}[h]
	\centering
	\subfloat[Performance]{{\includegraphics[width=0.47\linewidth]{figs/fairness_alpha}}}
	\hspace{0.05in}
    \subfloat[Fairness]{{\includegraphics[width=0.47\linewidth]{figs/fairness_std_alpha}}}
	\caption{[Simulation] Performance and fairness trade-off. A larger $\alpha$ helps \name approach \SRPT at the cost of worse fairness. }
	\label{fig:fairness_alpha}
\end{figure}

\subsection{Sensitivity Analysis}
\label{sec:sensitivity}

\subsubsection{Estimation errors}
\label{sec:estimation_err}

%Estimation errors are unavoidable.
Figure \ref{fig:analysis_err} evaluates the impacts of mis-estimations on the performance, where we vary the standard deviation of errors from 0 to 60\%.
Even though \DRFFIFO does not need the estimation, its performance is too bad so we compare \name with a stronger baseline \ESRP. %we compare the average completion time of \name to  \DRFFIFO 's.
Even with large estimation errors, \name still provides large improvements that are similar to \SRPT. 
%When the errors are small (under 20\%), the improvements are unchanged.
%When the errors are large (up to 43\%), \name is better than \DRFFIFO.
This highlights the value of incorporating (even noisy) estimation from our estimator. 

\begin{figure}[h]
	\centering
	\includegraphics[width=0.9\linewidth]{figs/analysis_err}
	\caption{[Simulation]  \name is robust to the estimation errors.}%(\DRFSJF is better with large errors because they can utilize more CPUs.) \todo{update the text accordingly.}}
	\label{fig:analysis_err}
\end{figure}

\subsubsection{Profiling overhead}
\label{sec:overhead}

Estimations require running profiling jobs with some overhead. Running a longer sample of a job provides better estimations with larger overhead. 
%The scales of estimation overheads may vary in practice.
Figure \ref{fig:analysis_overhead} evaluates the impacts of profiling overheads on performance. Recall in our experiments have profiling overheads at 3\%.
As \DRFFIFO does not require estimation, its performance is unchanged.% we compare the performance of \name with \DRFFIFO.
With large overheads, the performance of all other solutions including \name and \SRPT degrades. 
However, \name provides consistent, significant improvements compared to other baselines. 
%When the overheads are small (under ??\%), the performance improvement are significant.
In practice, for long and iterative machine learning jobs, it is reasonable to use small sampling jobs. 
We can also obtain the runtime estimation from users \cite{tetrisched, choosy}, which may lead to further improvements.

\begin{figure}[h]
	\centering	
	\includegraphics[width=0.9\linewidth]{figs/analysis_overhead_ext}
	\caption{[Simulation] Impacts of profiling overheads.}% must be small to have the small impacts on performance. Fortunately, sampling jobs can be much smaller than the full jobs as in our experiments. \todo{update the text accordingly.}}
	\label{fig:analysis_overhead}
\end{figure}

\subsubsection{Sensitivity to \name job subset size}
\label{sec:N_max}

Figure \ref{fig:analysis_N_max} shows the performance degradation of \name with smaller numbers of jobs used in the scheduling algorithm. There may be up to 200 jobs available for scheduling. As the figure shows, with a larger capacity of the system, more jobs should be included, but picking a small number already realizes most of the improvements. This allows \name to run at a much faster speed in practice. 
% We also have different capacity sizes.
% The maximum size of queued jobs of a user overtime is 97 (for capacity=20).
% The number of queued jobs linearly increases with the capacity.
% \name quickly converges at a small subset size that allows \name to overcome the computational bottleneck.

\begin{figure}[H]
	\centering
	{\includegraphics[width=0.9\linewidth]{figs/analysis_N_max_ext}}
	\caption{[Simulation] \name achieves most improvements by considering a small subset of available jobs. }
	\label{fig:analysis_N_max}
\end{figure}