\section{Online Interchangeable-Resource Allocation Problem}

\subsection{Problem Setting}

\desc{Interchangeability}

In the interchangeable resource allocation problem we considered, there are three resources of interests: CPU, GPU and memory (MEM).
Both CPU and GPU can be used for computational needs in an interchangeable way, MEM demand cannot be replaced by CPU or GPU.
This can be easily extended to include additional resources such as disk I/O, network I/O.

There are $N$ users in the systems. A user has jobs arriving over time.
Job $i$ arrives at time $t_i$, the online profiling tool (\S\ref{sec:res_estimation}) can create 2 configurations estimating the demand and processing time of job $i$ as following: $$
Q_i=
\begin{bmatrix} 
c_i & m_i^1 & p_i^1 \\
g_i & m_i^2 & p_i^2 
\end{bmatrix} $$ where $c_i, m_i^1, p_i^1$ represents the demand for CPU, memory and its estimated processing time from its CPU configuration; $g_i, m_i^2, p_i^2$ represents the demand for GPU, memory and its estimated processing time from its GPU configuration.

Let $C_c, C_g, C_m$ be capacities of CPU cores, GPUs and memory.
$A_c, A_g, A_m$ denotes current loads of the cluster on corresponding resources.

\paragraph{Problem statement} \emph{Given job configurations where jobs can interchange between CPUs and GPUs,
what is the best solution to allocate resources to the jobs such that it maximizes the performance and efficiency while maintains fairness?}

\todo{explain why's it is challenging?}

\subsection{Allocation Policy}

\begin{algorithm}[H]
\small
\caption{\name Online \name Scheduler}
\label{alg:greedy}
\begin{algorithmic}[1]
    \State $C,A:$ System capacity and current load
    \State $L:$ fairness score list of all users
    \State $Queue_i$: all waiting jobs of user $i$ at current time
    \State $U$: user set where the fairness scores of all users in $U$ fall in the lowest $\alpha$\% percentage
    \State $W$: all queueing jobs from set $U$.
    \State $S$: sorted list of all processing times from all jobs in $W$ (sorted increasingly)
    \Procedure{OnlineAlloc()}{}
    \While{$A_c< C_c || A_2 < C_2 ~\&\&~ A_3 < C_3$ }
    \State pick current value $p_n^m$ from list $S$		\Comment{Default is the first value in $S$}
    \If{ $m=2 \&\& A_2+ g_n \leq C_2 ~\&\&~ A_3+m_n^2 \leq C_3$ }  \Comment{GPU config}
    \State{\textsc{JobStart}($L,n$)} 
    \Comment{Place $n$ on GPU}
    \State{Update U,W,S}
    \ElsIf { $m=1 \&\& A_1+ c_n \leq C_1 ~\&\&~ A_3+m_n^1 \leq C_3$ }  \Comment{CPU config}
    \State{\textsc{JobStart}($L,n$)}  \Comment{Place $n$ on CPU}
    \State{Update U,W,S}
    \Else
    \State Move the next job;
    \EndIf
    \EndWhile
    \EndProcedure
    \\
    \Function{JobStart($L,job_j$)}{}  \Comment{Update score if job $j$ starts being processed}
    \State Remove job $j$ from list $W$
    \If {job $j$ is from user $i$}
    \If {$p_j^1 <  p_j^2$} 
    \If {job $j$ is placed on CPU}
    $Value(j) = \max(\frac{c_j}{C_1},\frac{m_j^1}{C_3})$
    \Else ~
    $Value(j) = \frac{p_j^1}{p_j^2}\max(\frac{c_j}{C_1},\frac{m_j^1}{C_3})$
    \EndIf
    \Else
    \If {job $j$ is placed on CPU}
    $Value(j) = \frac{p_j^2}{p_j^1} \max(\frac{g_j}{C_2},\frac{m_j^2}{C_3})$
    \Else ~
    $Value(j) = \max(\frac{g_j}{C_2},\frac{m_j^2}{C_3})$
    \EndIf
    \EndIf
    \EndIf
    \State $L(i) = L(i) + Value(j)$
    \EndFunction     
    \\
    \Function{JobFinish($L,job_j$)}{}  \Comment{Update score if job $j$ finishes}
    \If {job $j$ is from user $i$}
    \State $L(i) = L(i) - Value(j)$
    \EndIf
    \EndFunction
\end{algorithmic}
\end{algorithm}

The allocation policy needs to pack the jobs in the queue across candidate users.
The high level of the policy is as summarized in the following two steps:
(1) Pick a subset of users who receive less than others based on fairness scores.
(2) Schedule jobs of the user subset in to optimize their performance.
The pseudo-code of allocation policy is summarized in Algorithm \ref{alg:greedy}.

To maintain the fairness for step (1), we keep watch on the fairness score of each user.
%As sharing interachanable resources in space domain is not efficient, we need find another way to maintain fairness.
We define the fairness score as \textit{transformed dominant share}.
The idea is the following: Every user wishes his jobs to be finished as fast as possible.
However, that is not doable as some jobs have to sacrifice and take the suboptimal configurations for the performance of other jobs.
The system rewards them for taking a suboptimal solution and finish slower.
Here is how we define \textit{transformed dominant share} value $v(i)$ for a single job $i$. $v(i)$ will be calculated only when it is being processed.
From $$
	Q_i=
	\begin{bmatrix}
	c_i & m_i^1 & p_i^1 \\
	g_i & m_i^2 & p_i^2 
	\end{bmatrix} $$ simply compare $p_i^1$ and $p_i^2$ and check the placement of job $i$. \begin{itemize}
		\item If $p_i^1 <  p_i^2$, and job $i$ is on CPU, then $v(i) = \max(\frac{c_i}{C_1},\frac{m_i^1}{C_3})$.
		\item If $p_i^1 <  p_i^2$, and job $i$ is on GPU, then $v(i) =  (\frac{p_i^1}{p_i^2})^\mu \max(\frac{c_i}{C_1},\frac{m_i^1}{C_3})$.
		\item If $p_i^1 \geq  p_i^2$, and job $i$ is on CPU, then $v(i) =  (\frac{p_i^2}{p_i^1})\mu \max(\frac{g_i}{C_2},\frac{m_i^2}{C_3})$.
		\item If $p_i^1 \geq  p_i^2$, and job $i$ is on GPU, then $v(i) = \max(\frac{g_i}{C_2},\frac{m_i^2}{C_3})$.
	\end{itemize} 
where $\mu \geq 0 $ controls the fairness score when the job is not placed on the preferred resource.
The \textit{transformed dominant share} value of a user is the sum of all values from its jobs that are currently being processed.
\textit{Transformed dominant share} will shrink to DRF if there is only a single configuration.

In step (2), obtaining the optimal schedule is actually a nondeterministic polynomial (NP) time problem. 
We solve this step using heuristic.
A job $i$ has 2 completion times $p^1_i$ and $p^2_i$ on CPU and GPU respectively.
A set with $W$ jobs have $2|W|$ processing times.
We sort the processing times so that we select the job with shortest one to schedule first.
This heuristic is similar to the shortest remaining time first (SRPT) algorithm.