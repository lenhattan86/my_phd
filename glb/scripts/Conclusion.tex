\section{Concluding remarks}

In this paper, we present a systematic study of optimal energy procurement systems for geo-distributed data centers that utilize multi-timescale electricity markets. The contributions of this paper are three-fold: (i) designing algorithms for long-term electricity procurement in multi-timescale markets; (ii) analyzing  long-term prediction errors using real-world traces; and (iii)  empirically evaluating the benefits of our proposed procurement systems. In particular, we proposed two algorithms, PA and SGA, both of which save up to 44\% of the energy procurement cost compared to traditional algorithms that do not use long-term markets or geographical load balancing.  While SGA provably converges to an optimal solution, PA surprisingly achieves a cost that is nearly optimal with much less computing effort. 

There are a number of interesting directions for future research that can be motivated by our work. For example, generalizing our model to include more complicated forward contracts that procure energy that can be used over multiple time-slots is also another challenging problem. Integrating storage capabilities, e.g., batteries and/or thermal storage, into the energy procurement optimization of multi-timescale markets is another challenging direction. The further research on how to optimally utilize multi-timescale markets will have high potential to greatly impact the cost efficiency of Internet-scale services.

\phantomsection
\label{EndOfPaper}