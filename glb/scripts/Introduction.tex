\section{Introduction}

Data centers are becoming the largest and the fastest growing consumers of electricity in the United States. It is reported that US data centers consumed 91 billion kilowatt-hours (kWh) in 2013, which is more than twice of the electricity consumed by households in New York City (see \cite{whitney2014data}). In the same report, the electricity consumption of data centers is estimated to reach 140 billion kWh in 2020 due to the explosion of demand for cloud computing and other Internet-scale services. Global cloud providers such as Google and Amazon, who operate multiple data centers, spend billions of dollars annually on their electricity bills \cite{qureshi2009cutting}.


Multi-timescale electricity markets have been proposed to improve the efficiency of electricity markets \cite{cramton2007colombia}. Multi-timescale electricity markets encompass both {\em forward}  (futures) and {\em spot} (real-time) markets. While energy is procured at the time of consumption in a spot market, forward markets allow customers to buy electricity a day ahead or even several months ahead of when it is consumed. Forward electricity markets reduce the risk for both the supplier and consumer by reducing the quantity of energy trading in the real-time spot markets \cite{ausubel2010using}. Furthermore, purchasing electricity ahead of time can facilitate the expansion of renewable energy sources. For example, Google invested in purchasing renewable energy from renewable energy project developers for 20 years \cite{google2015datacenter}.


Utilizing multi-timescale markets has great potential for electricity cost savings for cloud providers who operate one or more data centers. There has been much recent work that exploits the variation of real-time electricity prices in the temporal and spatial dimensions to reduce the total electricity cost. For example, prior papers show how a cloud provider can exploit real-time electricity prices in multiple market locations and move the load to locations with a cheaper price \cite{qureshi2009cutting, chiaraviglio2011energy, rao2010minimizing}. Other papers exploit temporal variation in the real-time energy price and use energy storage to reduce electricity costs \cite{guo2011cutting, guo2013electricity,zheng2014exploiting}, i.e., the storage device is charged during the times when the electricity price is low and discharged when the price is high. However, while these works focus on traditional real-time markets, the potential of using multi-timescale markets for electricity cost reduction has not been well studied in the context of a cloud provider; this is the focus of the present paper.

In particular, using forward markets to lower the electricity cost of a cloud provider is challenging for multiple reasons. The optimal amount of electricity that a cloud provider should purchase in advance for a particular location depends on the workload, the onsite renewable generation, and the real-time electricity price at that location at the (future) time of delivery. But future workload, renewable generation, and real-time electricity price are not perfectly predictable and are subject to significant forecasting errors. Note that if the cloud provider is too conservative and buys too little from the forward market, any shortfall in electricity would need to be covered by purchasing it from the more expensive\footnote{In some cases, the prices in the forward markets might be (on average) higher than real-time prices. If so, instead of saving electricity expenditure, the cloud provider can participate in forward markets to reduce cost variations. Our model can be extended to handle either case.} real-time market. Likewise, if the cloud provider is too aggressive and buys too much from the forward market, any excess in purchased electricity will go wasted. Moreover, the ability of a cloud provider to move the load from one data center to another, possibly incurring a performance penalty that we characterize as the ``delay cost'', adds an additional level of complexity that needs to be optimized. In this work, we provide an optimization framework for tackling the aforementioned challenges.

\desc{Contributions of paper}

%\subsection{Our contributions}
Our contributions are three-fold.

\textbf{(1) Optimal algorithm development}. We develop two algorithms for a cloud provider with geo-distributed data centers to buy electricity in multi-timescale markets: one algorithm provides optimality guarantees, while the other is simpler, uses limited predictions but achieves near-optimal performance.
To develop the energy procurement system, we first model the problem of procuring electricity for geo-distributed data centers in multi-timescale markets in Section \ref{sec:model}. The system model is general and applicable to any global cloud provider with access to multi-timescale electricity markets. We focus on two-timescale markets that consist one long-term market and one real-time market, though our model and algorithms can be extended to handle multiple markets at various timescales. We present the characteristics of the objective functions and the optimal solution in Section \ref{sec:characterizingOptima}, which forms the theoretical basis for our algorithm design. The two algorithms that we design, prediction based algorithm (PA) and stochastic gradient based algorithm (SGA), are described in Section \ref{sec:AlgorithmDesign}. Both algorithms seek to minimize the total operating cost of the cloud provider across all data centers. While PA is simple and performs very well in practice, SGA provably achieves the optimal solution. 

\textbf{(2) Predictability analysis using real-world traces}. \diff{To provide the inputs for our energy procurement algorithms, we collect and analyze real world traces of PV generation, wind generation, electricity prices, and IT workload demand. A detailed data analysis of real-world traces of PV generation, wind generation, electricity prices, and workload, is presented in Section \ref{sec:dataAnalysis}. The data analysis not only enables us to evaluate our algorithms using real-world data but also provides insights into the nature of prediction errors. To procure electricity in forward markets, the energy procurement system needs to predict the renewable generation, workload, and electricity prices in real-time. Therefore, we focus on addressing the following questions. What do the distributions of prediction errors look like? How correlated are prediction errors in the spatial domain?}

\textbf{(3) Empirical evaluation}. We carry out a detailed empirical evaluation of our proposed energy procurement systems using real world traces. In Section \ref{sec:caseStudy}, we demonstrate that SGA can converge to the optimal solution in a small number of iterations. Moreover, we show that PA, our heuristic algorithm, surprisingly achieves a near-optimal solution. This is partially because the real-time optimization takes into consideration the trade-off between energy cost and delay cost, and is able to compensate for some prediction errors by redirection workloads. The proposed energy procurement systems are compared with other comparable energy procurement strategies to highlight their benefits. The impacts of renewable energy and prediction errors on the proposed systems are also presented.