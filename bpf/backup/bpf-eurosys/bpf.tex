% TEMPLATE for Usenix papers, specifically to meet requirements of
%  USENIX '05
% originally a template for producing IEEE-format articles using LaTeX.
%   written by Matthew Ward, CS Department, Worcester Polytechnic Institute.
% adapted by David Beazley for his excellent SWIG paper in Proceedings,
%   Tcl 96
% turned into a smartass generic template by De Clarke, with thanks to
%   both the above pioneers
% use at your own risk.  Complaints to /dev/null.
% make it two column with no page numbering, default is 10 point

% Munged by Fred Douglis <douglis@research.att.com> 10/97 to separate
% the .sty file from the LaTeX source template, so that people can
% more easily include the .sty file into an existing document.  Also
% changed to more closely follow the style guidelines as represented
% by the Word sample file. 

% Note that since 2010, USENIX does not require endnotes. If you want
% foot of page notes, don't include the endnotes package in the 
% usepackage command, below.

% This version uses the latex2e styles, not the very ancient 2.09 stuff.
\documentclass[preprint]{sigplanconf-eurosys}

\usepackage[T1]{fontenc}

\usepackage{epsfig,endnotes}
\usepackage{booktabs}

% shorten paper
%\usepackage[small,compact]{titlesec}
%
%\usepackage[bf]{caption}
\usepackage{amssymb}
\usepackage{amsmath}
\usepackage{amsfonts}
%\usepackage{amsthm}
% common
%\usepackage{cite} % citex error
\usepackage{url, xspace} 
\usepackage[small,bf]{caption}
%\usepackage{graphicx, cite, algorithm, algorithmic, color}
\usepackage{algorithm}
\usepackage{graphicx}
\usepackage{subfig}
\usepackage{bm}
%\usepackage{times} % => compiling error: Package textcomp Error: Symbol \textuparrow not provided by font family ppl in TS1 encoding. Default family used instead
%\usepackage{graphics}
%\usepackage{subfigure}
\usepackage{soul, color}
\soulregister\cite7
\soulregister\ref7
\usepackage[noend]{algpseudocode}

%\usepackage{subfigure}

\usepackage{hyperref}
\hypersetup{
  colorlinks=true,      % false: boxed links; true: colored links
  linkcolor=blue,       % color of internal links
  citecolor=magenta,    % color of links to bibliography
  filecolor=cyan,       % color of file links
  urlcolor=red          % color of external links
}

\usepackage{multirow}% http://ctan.org/pkg/multirow
\usepackage{hhline}% http://ctan.org/pkg/hhline

\usepackage{fnpos}

\newcommand{\reference}[2]{
	\ifthenelse{\boolean{isTechReport}}
	    {\ref{#1}} 
	    {#2}}

\newcommand{\myvec}[1]{\protect\overrightarrow{#1}}


%\newcommand{\desc}[1]{\hl{#1}}
\newcommand{\desc}[1]{}

%\newcommand{\delete}[1]{\st{#1}}
%\newcommand{\new}[1]{\hl{#1}}
\newcommand{\delete}[1]{}
\newcommand{\new}[1]{#1}

%  \addtolength{\textfloatsep}{-7mm}
%  \addtolength{\partopsep}{-1.3mm}
%  \addtolength{\columnsep}{-.3pc}
%  \renewcommand\floatpagefraction{.9}
%  \renewcommand\topfraction{.9}
%  \renewcommand\bottomfraction{.9}
%  \renewcommand\textfraction{.1}
%  \renewcommand{\baselinestretch}{0.98}
%  \makeatletter
%  \def\@listI{\leftmargin\leftmargini %% Added 22 Dec 87
%  \topsep 2pt plus 1pt minus 1pt\parsep 1.2pt plus 0.5pt minus 1pt
%  \itemsep \parsep}


\newcommand{\nhattan}[1]{\textcolor{green}{Tan says: #1}}
%\newcommand{\zhenhua}[1]{}

\newcommand{\zhenhua}[1]{\textcolor{blue}{Zhenhua says: #1}}
%\newcommand{\zhenhua}[1]{}

\newcommand{\mosharaf}[1]{\textcolor{blue}{Mosharaf says: #1}}
%\newcommand{\ramesh}[1]{}

\newcommand{\xiao}[1]{\textcolor{red}{Xiao says: #1}}

%\newcommand{\todo}[1]{\textcolor{red}{\textbf{(TODO: #1)}}}
\newcommand{\todo}[1]{}

\newcommand{\thoughts}[1]{\textcolor{blue}{(Potential: #1)}}


\newcommand{\diff}[1]{\textcolor{red}{#1}}

\newcommand{\argmin}{\arg\!\min}

\newtheorem{theorem}{Theorem}[section]
%\newtheorem{theorem}{Theorem}
\newtheorem{corollary}{Corollary}[theorem]
\newtheorem{lemma}[theorem]{Lemma}
\newtheorem{assumption}{Assumption}
\newtheorem{proof}{Proof}

\newcommand{\Exp}[1]{\mathbb{E}\left[#1\right]}
\newcommand{\ra}{\rightarrow}
\newcommand{\R}{\mathbb{R}}

%\newcommand{\Vector}[1]{\textit{\textbf{#1}}}
\newcommand{\Vector}[1]{\boldsymbol{#1}}

\newcommand{\ignore}[1]{}

\newenvironment{compactlist}{
 \begin{list}{{$\bullet$}}{
  \setlength\partopsep{0pt}
  \setlength\parskip{0pt}
  \setlength\parsep{0pt}
  \setlength\topsep{2pt}
  \setlength\itemsep{4pt}
  \setlength{\itemindent}{\leftmargin}
  \setlength{\leftmargin}{0pt}
 }
}{
 \end{list}
}

\newenvironment{denseitemize}{
\begin{itemize}[topsep=2.5pt, partopsep=0pt, leftmargin=1.5em]
  \setlength{\itemsep}{2.5pt}
  \setlength{\parskip}{0pt}
  \setlength{\parsep}{0pt}
}{\end{itemize}}

\newenvironment{denseenum}{
\begin{enumerate}[topsep=2.5pt, partopsep=0pt, leftmargin=1.5em]
  \setlength{\itemsep}{2.5pt}
  \setlength{\parskip}{0pt}
  \setlength{\parsep}{0pt}
}{\end{enumerate}}

\newenvironment{densedesc}{
\begin{description}[topsep=2.5pt, partopsep=0pt, leftmargin=1.5em]
  \setlength{\itemsep}{2.5pt}
  \setlength{\parskip}{0pt}
  \setlength{\parsep}{0pt}
}{\end{description}}

\def\name{BoPF\xspace}

\def\bursty{Bursty}
\def\batch{Batch\xspace}

\def\ala{{\`{a} la}~}
\def\ie{{i.e.}}
\def\eg{{e.g.}}
\def\etal{{et al.}~}
\def\wrt{{w.r.t.}~}
\def\viz{viz.~}
\def\vs{vs.~}
\def\etc{etc.}


\makeatletter
\def\BState{\State\hskip-\ALG@thistlm}
\makeatother


\newcommand{\cL}{{\cal L}}
\def\batchq{TQ\xspace}
\def\burstq{LQ\xspace}


\title{\Large \bf \name: Mitigating the Burstiness-Fairness Tradeoff in Multi-resource Clusters -- Technical Report}
%\authorinfo{Tan N. Le$^{1,2}$, Xiao Sun$^{1}$, Mosharaf Chowdhury$^{3}$, and Zhenhua Liu$^{1}$ \\
%}
%{$^1$Stony Brook University, $^2$SUNY Korea, $^3$ Universify of Michigan}{}
\authorinfo{}
{}{}

\hypersetup{draft} % temporaly fix compiling error for BIBTEX

\begin{document}
\maketitle

\subsection*{Abstract}
Simultaneously supporting latency- and throughout-sensitive workloads in a shared environment is a classic networking challenge that is becoming increasingly more common in big data clusters. 
Despite many advances, existing cluster schedulers force the same performance goal -- fairness in most cases -- on all jobs. 
Latency-sensitive jobs suffer, while throughput-sensitive ones thrive. 
Using prioritization does the opposite: it opens up a path for latency-sensitive jobs to dominate. 
In this paper, we tackle the challenges in supporting both short-term performance and long-term fairness simultaneously with high resource utilization by proposing Bounded Priority Fairness (\name). 
\name provides short-term resource guarantees to latency-sensitive jobs and maintains the long-term fairness for throughput-sensitive jobs. 
The key idea is ``bounded'' priority for latency-sensitive jobs; meaning, if bursts are not too large to hurt the long-term fairness, they are given higher priority so jobs can be completed as quickly as possible. \name is the first scheduler that can provide long-term fairness, burst guarantee, and Pareto efficiency in a strategyproof manner for multi-resource scheduling. 
Additional treatment is included for uncertainties on the burst sizes.
Deployments and large-scale simulations show that \name closely approximates the performance of Strict Priority as well as the fairness characteristics of DRF. 
In deployments, \name speeds up latency-sensitive jobs by $5.38\times$ compared to DRF, while still maintaining long-term fairness. 
In the meantime, \name improves the average completion times of throughput-sensitive jobs by up to $3.05\times$ compared to Strict Priority.

%\nhattan{Do we need to properties of BPF here?}

%\todo{Write stuff about our proposal and result.}

% Popular resource allocations such like DRF aim at instantaneous fairness, even though applications and jobs care about long-term metrics, e.g., completion time.
% This gap may result in high variation in performance, which hurts cloud computing users and results in low resource utilization in static allocations.
% In this work, we aim to achieve both performance and fairness with high resource utilization by designing \name.
% \name is built upon the key observation that cloud applications have quite distinct characteristics and performance requirements.
% Streaming jobs and interactive jobs such as Spark applications have multiple arrivals of (small) batch jobs, and once a job arrives, the goal is to finish it as quickly as possible.
% On the other hand, many other jobs have much larger sizes, and instead of response time, the long-term average resources allocated matter.
% This gives us a promising opportunity to prioritize streaming jobs and interactive jobs in the short term, and ensure fairness in the long term.
% Instead of simple prioritization, where streaming jobs cannot take arbitrary amounts of resources, we adopted bounded prioritization.
% Deployments and large-scale simulations show that \name closely approximates the performance of Strict Priority, and the fairness of DRF \cite{drf}.
% In the deployments, \name speed up latency-sensitive jobs 5.38x (totally 9 queues in a cluster) faster than DRF and still maintains the long-term fairness to protect the throughput-sensitive jobs.


\section{Introduction}

Cloud computing infrastructures are increasingly being shared between diverse workloads with heterogeneous requests. 
In particular, throughput-sensitive batch processing systems \cite{mapreduce, dryad, graphlab, tez} are often complemented by latency-sensitive interactive analytics \cite{spark, blinkdb, presto} and online stream processing systems \cite{spark-streaming, trident, millwheel, naiad, flink}.
% 
Simultaneously supporting these workloads is a balancing act between distinct performance metrics.\new{For example, Figure~\ref{fig:queues} shows that the cluster schedules the mix of jobs from a throughput-sensitive queue (TQ) and a latency-sensitive queue (LQ).
Batch processing workloads such as indexing \cite{mapreduce} and log processing \cite{hadoop, spark} may submit hours-long large jobs via TQ. 
The \emph{average} amount of resources received over a certain period is critical for its progress.
On the LQ, interactive \cite{spark, presto} and online streaming \cite{spark-streaming, trident} workloads respectively submit on-demand and periodic smaller jobs. 
Therefore, receiving enough resources immediately for an LQ job is more important than the average resources rece ived over longer time intervals.} 

\begin{figure}[!t]
  \centering
  \includegraphics[width=0.8\columnwidth]{fig/queues.pdf}%
  \caption{\new{Users and automated processes submit throughput-sensitive (TQ) and latency-sensitive (LQ) to the same cluster.}}%
  \label{fig:queues}
 	\vspace{-0.1in}
\end{figure}

To address the diverse goals, today's schedulers are becoming more and more complex.
They are multi-resource \cite{drf, tetris, multiresource-mungchiang, orchestra, pacman}, DAG-aware \cite{aalo, tetris, spark}, and allow a variety of constraints \cite{late, quincy, mantri, choosy, delay-scheduling}. 
Given all these inputs, they optimize for objectives such as fairness \cite{drf, jaffe-maxmin, drfq, hdrf}, performance \cite{sjf}, efficiency \cite{tetris}, or different combinations of the three \cite{carbyne, graphene}.
However, all existing schedulers have one shortcoming in common: \emph{they force the same performance goal on all jobs while jobs may have distinct goals}, and therefore fail to provide performance guarantee in the presence of multiple types of workloads with different performance metrics. 
In fact, the performance of existing schedulers can be \emph{arbitrarily bad} for some workloads.

\begin{figure}[!t]
	\centering
	\includegraphics[width=0.3\linewidth]{fig/b1_mov_legend} 
	\\
	\subfloat[DRF ensures instantaneous fairness, but increases the completion times of latency-sensitive jobs.]
  {\includegraphics[width=0.85\linewidth]{fig/b1_mov_DRF_BB}\label{fig:motiv-DRF}}
	%\subfloat[DRF -- The completion time of 4 \burstq jobs are  202, 178, 715, and 719 seconds, respectively.]{\includegraphics[width=1.0\linewidth]{fig/b1_mov_DRF_BB}\label{fig:motiv-DRF}}
	\\
	\subfloat[SP decreases completion times of latency-sensitive jobs, but throughput-sensitive batch jobs do not receive their fair shares.]
  {\includegraphics[width=0.85\linewidth]{fig/b1_mov_Strict_BB}\label{fig:motiv-Strict}}
	%\subfloat[Strict Prority (SP)-- The completion time of 4 \burstq jobs are  124, 121, 430, and 405 seconds, respectively.]{\includegraphics[width=1.0\linewidth]{fig/b1_mov_Strict_BB}\label{fig:motiv-Strict}	}
	\\
	\subfloat[The ideal solution allows first two latency-sensitive jobs to finish as quickly as possible, but protects batch jobs from latter \burstq jobs by ensuring long-term fairness.]
  {\includegraphics[width=0.85\linewidth]{fig/b1_mov_Optimum_BB}\label{fig:motiv-optimal}}	
	\caption{Need for bounded priority and long-term fairness in a shared multi-resource cluster with latency-sensitive (\burstq: blue/dark) and throughput-sensitive (\batchq: orange/light) jobs. \new{The blank part on the top is due to resource fragmentation and overheads in Apache YARN.}
    Although we focus only on memory allocations here, similar observations hold in multi-resource scenarios. \todo{put the memory (TB) instead of (TB) on the figures.}}
%    \todo{We should also have a figure showing the arrivals of LQ A.}
%    \mosharaf{Something is wrong with the third figure. Total area of blue should be the same regardless of algorithm.} 
%    \zhenhua{I didn't see the problem. The area for each arrival seems the same.}}
    %In DRF, it is unfair for the first two jobs to slow down their progress although they require very small resources. From 0 to 1400 seconds, Strict speeds up the delay-sensitive jobs from \burstq  but it later makes \batchq starving of resources because of the large jobs. \todo{change \burstq  and \batchq to \burstq  and \batchq}}
	\label{fig:motiv_ex}
		\vspace{-0.1in}
\end{figure}

Consider the simple example in Figure~\ref{fig:motiv_ex} that illustrates the inefficiencies of existing resource allocation policies, in particular, DRF~\cite{drf} and Strict Priority (SP) \cite{strict_priority} for coexisting workloads with different requirements. 
In this example, we run memory-bound jobs in 
\delete{Tez atop Apache YARN}
within a cluster of 40 nodes, where each node has 32 CPU cores and 64 GB RAM.
% 
There are two queues in the system, where each queue contains a number of jobs with the same performance goals.  
\new{Apache Hadoop YARN \cite{hadoop-fair-scheduler} is set up to manage the resource allocation among the queues.}
The first queue is for Spark streaming, which submits \delete{a number of MapReduce jobs }\new{a MapReduce job} every 10 minutes. 
We call it latency-sensitive queue (\burstq) because it aims to finish the jobs as quickly as possible.
The second queue is a throughout-sensitive batch-job queue (\batchq) formed by jobs generated from the BigBench workload~\cite{bbench} and queued up at the beginning. \batchq cares more about its long-term averaged resources received, e.g., every 10 minutes.
\todo{remove the TQ and LQ definition as they are defined in the Figure 1.}
For simplicity of exposition, all jobs are memory-bound. 
%While \burstq  tries to complete \emph{each job} as quickly as it arrives, \batchq cares more about its long-term averaged resources received, e.g., every 10 minutes.
% 
We consider two extreme classes of policies -- priority-based and fairness-based allocation -- in this example, where the former is optimized for latency and the latter for fairness.
% There are many other policies in between such as service curve and network calculus in general.
% Service curve and network calculus are often too conservative; \ie, they only admit a very small number of queues \cite{}.
Specifically, we focus on the inefficiency of DRF and SP. 
The memory resource consumption
%\footnote{The blank part on the top is due to resource fragmentation and overheads in Apache YARN.}
under these two policies is depicted in Figures~\ref{fig:motiv-DRF} and~\ref{fig:motiv-Strict}, respectively. 
We defer the discussion of other policies to Section~\ref{sec:property-analysis}. 
%\todo{We should add numbers about the completion time under two }
%\mosharaf{We should probably remove TQ and LQ names and just say latency- and throughput-sensitive jobs.}
%\zhenhua{Agreed. }

SP gives \burstq the whole cluster's resources (high priority) whenever it has jobs to run; hence, it provides the lowest possible response time. 
For the first two arrivals, the average response time is 130 seconds. 
A detrimental side effect of SP, however, is that there is no resource isolation -- \batchq jobs may not receive any resources at all! 
In particular, \burstq has incentives to increase its arrivals -- e.g., for more accurate sampling and more iterations in training neural networks -- without any punishment. 
As it does so from the third job arrival, \batchq no longer receives its fair share. 
In the worst case, \burstq can take all the system resources and \emph{starve} \batchq. 
In summary, SP provides the best response time for \burstq , but no isolation protection for \batchq at all.
In addition, SP is incapable of handling multiple {\burstq}s. 
% \zhenhua{Actually, when \burstq  increases its burst size, its response time goes worse!}

In contrast, DRF enforces \emph{instantaneous} fair allocation of resources at all times. 
During the burst of \burstq , \burstq  and \batchq share the bottleneck resource (memory) evenly until the jobs from \burstq  complete; then \batchq gets all resources before the next burst of \burstq . 
Clearly, \batchq is happy at the cost of longer completion times of \burstq 's jobs, whose response time increases by 1.6 times. 
In short, DRF provides the best isolation protection for \batchq, but no performance consideration for \burstq . 
When there are many {\batchq}s, the response time of \burstq  can be very large. 

%Therefore, it is natural to ask if we can achieve the best response time of {\burstq}s and fairness for {\batchq}s simultaneously. 
Clearly, it is impossible to achieve the best response time under \emph{instantaneous} fairness, which fully decides the allocation. 
In other words, there is a hard tradeoff between providing instantaneous fairness for {\batchq}s and minimizing the response time of {\burstq}s.
Consequently, we aim to answer the following fundamental question in this paper: \emph{how well can we simultaneously accommodate multiple classes of workloads with performance guarantees, in particular, isolation protection for {\batchq}s and low response times for {\burstq}s}? 

\begin{figure}[!t]
  \centering
  \includegraphics[width=0.85\linewidth]{fig/Design-Space.pdf}%
  \caption{{\name} in the cluster scheduling design space.}%
  \label{fig:design-space}
  	\vspace{-0.2in}
\end{figure}

We answer this question by designing \name: the first multi-resource scheduler that achieves both isolation protection for {\batchq}s in terms of \emph{long-term} fairness and response time guarantees for {\burstq}s, and is strategyproof. 
It is simple to implement and provides significant performance improvements even in the presence of uncertainties. 
The key idea is ``bounded'' priority for {\burstq}s: as long as the burst is not too large to hurt the long-term fair share of {\batchq}s, they are given higher priority so jobs can be completed as quickly as possible. 
Figure~\ref{fig:design-space} shows \name in the context of cluster scheduling landscape. 
%There are several challenges, and by solving them 
We make the following contributions.

\paragraph{Algorithm design.} We develop \name with the rigorously proven properties of strategyproofness, short-term bursts, long-term fairness, and high system utilization (\S\ref{sec:approach}). When {\burstq}s have different demands for each arrival, we further design mechanism to handle the uncertainties.

%\paragraph{Handling uncertainties.} We develop a simple yet efficient method to allow customers with significant estimation errors of their burst sizes to request the right amount of resources to satisfy their performance goals. We also highlight the difference between single-resource allocation and multi-resource allocation in the presence of estimation errors. In particular, we prove that the resource demands in the multi-resource scenario is always higher than those if resources are allocated individually unless there is no estimation error. This is true even when errors on different resources are independent. In addition, larger estimation errors lead to higher resource demands.
%\mosharaf{Didn't understand the last couple sentences, especially the phrase ``risk premium''.}
%\zhenhua{I adjusted. Let me know if it works.}

\paragraph{Design and implementation.} 
We have implemented \name on Apache YARN \cite{yarn} (\S\ref{sec:impl}).
Any framework that runs on YARN can take advantage of \name.
The \name scheduler is implemented as a new scheduler in Resource Manager that runs on the master node.
The scheduling overheads for admitting queues or allocating resources are negligibly less than 1 ms for 20,000 queues.


%
%We implement the system. One challenge is the delay in serving the bursts of {\burstq}s. \todo{Feel free to add more.}
%\mosharaf{This is probably not big enough to highlight here. Should be merged with the eval.}
%\zhenhua{That's fine. Just want to make sure it's mentioned.}

\paragraph{Evaluation based on both testbed experiments and large-scale simulations.}
In deployments, \name significantly provides up to $5.38\times$ lower completion times for \burstq jobs than DRF, while maintaining the same long-term fairness (\S\ref{sec:testbed}). 
At the same time, \name provides up to $3.05\times$ more fair allocation to {\batchq} jobs compared to SP.
%Furthermore, \name performs well even in the presence of demand variations and moderate misestimations (\S\ref{sec:sensitivity_analysis}).%\todo{sensitivity analysis regarding arrival time?}.
%\mosharaf{Not sure about talking about the preemption business here.}
%\zhenhua{Those are old texts. Feel free to edit.}

%
%\zhenhua{I'm not sure when to put the ideal allocation}
%\mosharaf{Following can be dropped.}
%The ideal allocation we target at is depicted in Figure~(\ref{fig:motiv-optimal}).
%Despite the negative result, the good news is that if we relax the instantaneous fairness to long-term fairness, both properties can be achieved at the same time. In fact, long-term fairness is good enough to provide the isolation protection to {\batchq}s. This is depicted in Figure~\ref{fig:motiv-optimal} as the ideal allocation. 
%The key idea is ``bounded'' priority for {\burstq}s: as long as the burst is not too large to hurt the fair share of {\batchq}s, they are given higher priority so jobs can be completed as quickly as possible. 
%In particular, before 1,400 seconds, \burstq 's bursts are small, so it gets higher priority, which is similar to SP. 
%After \burstq  increases its demand, only a fraction of its demand can be satisfied with the entire system's resources. Then it has to give resources back to \batchq to ensure the long-term fairness.
%
%
%=======old text=======
%
%
%Nowadays, there are multiple types of workloads with heterogeneous objectives. 
%(1) (Periodic) bursty workloads such as Spark streaming, scheduled tasks and routines. Their main goal is to minimize response time in seconds or even milliseconds.
%(2) Batch jobs. Long-term throughput measured in minutes or hours.
%(3) Interactive workloads. Non-periodic but can depend on response time, to minimize response time. Service curve is more suitable. 
%
%\thoughts{Is there a way to unify different objectives? Initial thoughts: demand can be represented by $d(t)$, where $t$ is discrete for presentation simplicity but can be easily extended to continuous case. Then $d(t)$ is nonzero at  periodic timeslots for (1), at one timeslot for (2), and at random timeslots for (3). Performance can be measured at each period for (1), after a sufficiently long interval for (2) as progress, and between consecutive arrivals for (3). Even though these workloads look different, there seems to be a way to unify them.}
%
%Under the cloud computing regime, either private or public, these workloads may share the same underlying physical infrastructure. 
%The resource contention among workloads leads to scheduling and resource allocation.
%In nowadays settings, there are multiple queues, each of which consists of a number of jobs. Each job can have multiple tasks potentially with interdependency.
%Jobs of the same queue belong to the same workload, and therefore have the same objective\footnote{A related concept is user. A user may have multiple queues, e.g., for different applications.}.
%Each queue has a weight and needs isolation guarantee. 
%Jobs may be large, e.g., for throughput-sensitive workloads. In this case, their progress can be measured by the number of tasks served within a time interval. \zhenhua{We need to clarify this to be consistent with the numerics.}
%\mosharaf{Is it possible to talk about things without introducing queues first? Then introducing queues. Btw, we should look up the Jockey paper from EuroSys 2012 carefully.}
%
%Despite the heterogeneity, start-of-the-art schedulers force the same performance goal on all workloads, which incurs inefficiency.
%
%Consider the example of a system serving both throughput-sensitive and latency-sensitive workloads. In this example, the number of tasks completed within a time interval is used to capture the progress of a throughout-sensitive batch-job queue (\batchq) \cite{tetris, carbyne, mantri, late}.
%Interactive sessions and streaming applications form latency-sensitive queues (\burstq): each \burstq is a sequence of small jobs following an ON-OFF pattern. \zhenhua{It's not really a ON-OFF pattern. It's more like a periodic arrival with deadlines...}
%For these jobs \cite{spark-streaming, spark, splunk-analysis, presto}, individual completion times or latencies are far more important than the throughput of the \burstq. 
%\mosharaf{I agree. We should get rid of this ON-OFF part. Another thing to do would be getting rid of asking for period lengths from users.}
%
%\todo{add an example to show the problem.}
%
%There are two classes of schedulers in literature, but neither of them solved the problem completely.
%\mosharaf{There is a third category that focuses on increasing utilization.}
%
%Priority-based: good for \burstq, while \batchq can be starved
%
%Fairness-based such as DRF: good for \batchq, while response time of \burstq can be arbitrarily worse compared to the optimal solution (one \burstq, $n$ {\batchq}s, $n$ goes to infinity)
%
%Other solutions: BVT, Carbyne \zhenhua{Any others? We need to discuss all of them, perhaps in the motivation section}
%\mosharaf{We may need to update the taxonomy as well.}
%
%Therefore, we ponder a fundamental question: \emph{can we enable latency-sensitive {\burstq}s and throughput-sensitive {\batchq}s to coexist, where {\burstq}s are permitted short-term resource bursts while ensuring long-term fairness between the two, as well as maximizing {\burstq}s served with resource guarantees?}
%
%In this paper, we first show that there is a hard tradeoff between instantaneous fairness and short-term resource bursts. 
%
%Then we relax the fairness notion from instantaneous one to long-term one and the key opportunity emerges: {\batchq}s do not care about allocated resources in the short term, as long as the long-term averaged resource share remains the same. This temporal flexibility allows us to accommodate bursts of {\burstq}s without hurting {\batchq}s. 
%
%However, naive admission control may result in very few {\burstq}s admitted with resource guarantees, leading to suboptimal performance for {\burstq}s even with a high system utilization.
%
%In addition, there are challenges for multi-resource allocation \zhenhua{is this trivial?} and handling the uncertainties.
%
%%\textcolor{blue}{
%%Additionally, how one reasons about system utilization becomes more complicated in the presence of both {\burstq}s and {\batchq}s. 
%%While it is relatively easy to fill the cluster with many batch jobs to achieve high utilization, {\burstq}s often have higher value and are harder to accommodate with resource guarantees. 
%%Compared to a simple solution that rejects most of {\burstq}s and/or serves them with best efforts, system operators are likely to prefer to maximize the {\burstq}s served with resource guarantees, which leads to more predictable performance. %We call the resources allocated to {\burstq}s with resource guarantees {\burstq} utilization.
%%%}
%%}
%%
%%
%%The challenges are \zhenhua{we should add some key messages from theoretical results. My current thoughts are: BG and LF seem not very challenging given the opportunity we discussed in the last paragraph, so we should focus on strategyproofness and (LQ) utilization. Xiao, as you have some concerns regarding the strategyproofness, can you write down your thoughts here? If BPF in the current form is strategyproof, then we should highlight some other alternatives are not, similar to our HUG paper. Otherwise, we should improve BPF and use the current version as the example. I can write something along our NSF proposal regarding the tradeoff between progress and utilization. I looked into our HUG paper, and it seems the intro has quite a few strong points, such as DRF's utilization can be arbitrarily low, naive work conservation can lead to worse utilization, etc. Meanwhile, the elastic demand is very clear. We should try to do something similar. }
%%
%%
%%
%%\textcolor{blue}{
%%The need to separate throughput- and latency-sensitive job queues in big data clusters is akin to the classic networking problem of supporting throughput-sensitive flows and latency-sensitive realtime communication on network access links \cite{cbq, intserv-hierarchy, hfsc, cruz-sc}.
%%The key difference between the two is the multi-resource nature of cluster scheduling and challenges arising from it. 
%%We aim for a solution that can be applied to multi-resource clusters.
%%}
%
%\textcolor{blue}{
%%\todo{five properties, but some are missing below.}
%In this paper, we first explore opportunities and challenges in achieving four desired properties in this context (\S\ref{sec:motivation}): (i) allowing {\burstq}s to enjoy short-term high resource usage during their ON periods to minimize \emph{individual} job response times; (ii) maintaining the same \emph{long-term} fairness as existing multi-resource fair allocation policies by preventing arbitrarily large bursts; (iii) maximizing system utilization and {\burstq}s served with resource guarantees; %\new{(iv) It is better off sharing than using static sharing}. 
%and (iv) the policy needs to be strategyproof because otherwise {\burstq}s may lie about their demands to gain more resources. 
%}
%
%\subsection*{Contributions}
%
%\paragraph{Algorithm development.} properties with rigorous proofs: strategyproofness, short-term bursts, long-term fairness, system utilization
%
%\paragraph{Design and implementation.} any challenges here? The slow start problem that Tan is currently working on is one potential.
%
%\paragraph{Evaluation based on both testbed experiments and large-scale simulations.}
%In deployments, \name significantly provides up to $5.38\times$ (totally 9 queues in a cluster) faster performance with smaller variation for \burstq jobs than DRF, while maintaining the fairness among the queues (\S\ref{sec:testbed}). 
%\name protects the {\batchq} jobs up to $3.05\times$ compared to SP.
%Furthermore, \name performs well even in the presence of moderate misestimations of task demands and without preemption allowed (\S\ref{sec:sensitivity_analysis})\todo{sensitivity analysis regarding arrival time?}.
%
%\section{Motivation}
%
%\mosharaf{If we could talk about some numbers that'd be good. Of course, we don't have any real traces ourselves. One possible option is going through existing works and see if they mention any high-level characteristics that we can use.}
%\zhenhua{I agree. While searching for existing work, we can start from some simple examples as we did in the HUG paper, which is well received.}
%
%\subsection{Problem Formulation}
%
%\subsection{Desired Properties}
%
%\subsection{Challenges and Inefficiencies of Existing
%Allocation Policies}
%
%SP
%
%DRF
%
%BVT
%
%Carbyne
%
%others?
%
%\subsection{Hard tradeoff between instantaneous fairness and bursty accommodations}
%
%\subsection{Naive admission control results in few {\burstq}s admitted with performance guarantee}
%
%\subsection{Estimation errors matter}
%
%
%\section{\name: analytical results}
%
%The \name algorithm
%
%long-term fairness
%
%burst guarantee
%
%Pareto efficiency
%
%strategyproofness
%
%\section{Stochastic optimization to handle the uncertainties}
%\zhenhua{Xiao is working on this}
%
%Model
%
%Algorithm
%
%
%\section{System design}
%
%Reducing overheads from creating containers. \zhenhua{Tan is looking into this}
%
%\section{Evaluations}
%
%\zhenhua{Tan, can you add the current figure list here?}
%
%\todo{Everyone, please add to the figure list addition figures that may help.}
%
%
%We discuss related work in Section~\ref{sec:related}.

%\mosharaf{A list of three contributions? Problem formulation, Algorithm and its properties, and full system evaluation and thorough eval.}


\section{Motivation}
\label{sec:motivation}

% \todo{make sure we use queue(s) throughout the paper instead of user(s)}
%In this section, we demonstrate the benefits of temporal co-scheduling of queues with heterogeneous performance metrics using an example to compare against existing solutions (Section~\ref{sec:ex}). 
%Next, we explore the desired properties and the design space for such a solution (Section~\ref{sec:desire}), followed by the discussion of challenges in achieving high \burstq utilization by admitting more {\burstq}s with resource guarantee.
% Our model and solution approach apply to a general class of resource allocation problems including those in the networking community (Section~\ref{sec:networking}) \todo{add}.

%While hard guarantee and instantaneous fairness are with fundamental tradeoffs, (Section~\ref{sec:desire})  and.  and inefficiencies in existing schedulers (Section~\ref{sec:desire}), and discusses the challenges associated (Section~\ref{sec:challenge}).


\subsection{Benefits of Temporal Co-scheduling}
\label{sec:ex}


%\begin{figure}[!t]
%	\centering
%	\includegraphics[width=0.3\linewidth]{fig/b1_mov_legend} 
%	\\
%	\subfloat[DRF: While instantaneous fairness is enforced, the completion time of \burstq jobs is much higher than that under SP.]{\includegraphics[width=0.8\linewidth]{fig/b1_mov_DRF_BB}\label{fig:motiv-DRF}}
%	%\subfloat[DRF -- The completion time of 4 \burstq jobs are  202, 178, 715, and 719 seconds, respectively.]{\includegraphics[width=1.0\linewidth]{fig/b1_mov_DRF_BB}\label{fig:motiv-DRF}}
%	\\
%	\subfloat[Strict Priority (SP): While the completion time of \burstq jobs are reduced, \batchq does not receive its fair share after \burstq increases its demand after 1,400 seconds.]{\includegraphics[width=0.8\linewidth]{fig/b1_mov_Strict_BB}\label{fig:motiv-Strict}}
%	%\subfloat[Strict Prority (SP)-- The completion time of 4 \burstq jobs are  124, 121, 430, and 405 seconds, respectively.]{\includegraphics[width=1.0\linewidth]{fig/b1_mov_Strict_BB}\label{fig:motiv-Strict}	}
%	\\
%	\subfloat[Ideal: The first two \burstq jobs finish as quickly as possible. The latter jobs cannot use more resources than \burstq's fair share.]{\includegraphics[width=0.8\linewidth]{fig/b1_mov_Optimum_BB}\label{fig:motiv-optimal}}	
%	\caption{Need for bounded priority and long-term fairness in a shared multi-resource cluster.
%    Although we only focus on memory allocations here, similar observations hold for other resources.}%In DRF, it is unfair for the first two jobs to slow down their progress although they require very small resources. From 0 to 1400 seconds, Strict speeds up the delay-sensitive jobs from \burstq but it later makes \batchq starving of resources because of the large jobs. \todo{change \burstq and \batchq to \burstq and \batchq}}
%	\label{fig:motiv_ex}
%\end{figure}

Consider the example in Figure~\ref{fig:motiv_ex} again. Recall that SP and DRF are two extreme cases in trading off performance and fairness: SP provides the best performance (for {\burstq}s) with no fairness consideration (for {\batchq}s); DRF ensures the best isolation (for {\batchq}s) with poor performance (of {\burstq}s). 
However, it is still possible for {\burstq}s and {\batchq}s to share the cluster by thoughtful co-scheduling over time.

The ideal allocation is depicted in Figure~\ref{fig:motiv-optimal}. 
The key idea is ``bounded'' priority for {\burstq}s as we discuss in the previous section. 
In particular, before 1,400 seconds, \burstq's bursts are small, so it gets higher priority, which is similar to SP. 
After \burstq increases its demand, only a fraction of its demand can be satisfied with the entire system's resources. Then it has to give resources back to \batchq to ensure the long-term fairness.

%in IQs unless we set very high weights to IQs. Even in this case, there are at least two problems: (a) BQ starvation, (b) still no performance guarantee with multiple IQs because giving them all high weights does not work.

%DRF/HUG: similar to FS as only focusing on instantaneous allocation.

%Strict priority (SP): BQ starvations
%
%\zhenhua{We should discuss other important schedulers. Mosharaf, could you please add some?}
%
%Figure: DRF/SP/DRFw/\name with small interactive queue: SP $>$ DRFw $>>$ DRF (\name performs similar to SP, better than DRFw, and much better than DRF)
%
%Figure: DRF/SP/DRFw/\name with small interactive queue: DRF $>>$ DRFw $>$ SP (\name performs similar to DRF, better than DRFw, and much better than SP)




%\emph{Example Setup} 
%
%\begin{itemize}
%\item Framework: Tez atop YARN.
%\item Assumption: All the jobs are memory bound jobs so that it is equivalent to single resource allocation.
%\item cluster: 40 nodes. each node has 64 GB RAM.
%\item \burstq submits delay-sensitive jobs. \burstq submits 4 Map Reduce jobs every 600 secs. A Map task <14GB, 1 cpu> resource takes 25 secs and a Reduce task <20GB, 1 cpu> takes 2 secs.
%	\begin{itemize}
%	\item 2 first jobs have 480 Map tasks and 80 Reduce tasks. 
%	\item The 3rd and 4nd jobs have 1920 Map tasks and 320 Reduce tasks. 
%	\end{itemize} 
%\item \batchq submits the jobs generated from BigBench workload. The \batchq's jobs are queued up at the beginning.
%\end{itemize}

%In Figure \ref{fig:mov_example}, Strict allows \burstq to speeds up his jobs to as soon as possible. However, Strict hurts \batchq a lot when the delay-sensitive jobs are too large (from 1000).





%\subsection{Just to organize our own thoughts}


\subsection{Desired Properties}
\label{sec:desire}

% \begin{figure}
% \centering
% \includegraphics[scale=0.35]{fig/tradeoff}
% \caption{Design space -- \name provides {\burstq}s with bounded priority and maintains long-term fairness. \todo{I think we achieve fairness not bounded fairness.}}
% \label{fig:tradeoff}
% \end{figure}

We restrict our attention in this paper to the following, important properties: burst guarantee for {\burstq}s, long-term fairness for {\batchq}s, strategyproofness, and Pareto efficiency to improve cluster utilization.

\textbf{Burst guarantee (BG)} provides performance guarantee for {\burstq}s by allocating guaranteed amount of resources during their bursts. 
In particular, an {\burstq} requests its minimum required resources for its bursts to satisfy its service level agreements, e.g., percentiles of response time. 
%\xiao{Another things is that, I feel burst guarantee is promised by our admission control procedure, while other policies do not have such step. It's kind of unfair to compare to others using this property.}

%the key to provide service for latency-sensitive jobs, especially in latency-sensitive interactive analytics and online stream processing systems. In BPF, we provide as many LQ queues with burst guarantee as possible, given the long-term fairness ensured for both LQ and TQ queues.

\textbf{Long-term fairness (LF)} provides every queue in the system the same amount of resources over a (long) period, e.g., 10 minutes.
Overall, it ensures that {\batchq}s progress no slower than any {\burstq} in the long run. LF implies sharing incentive, which requires that each queue should be better off sharing the cluster, than exclusively using its own static share of the cluster. 
If there are $n$ queues, each queue cannot exceed $\frac{1}{n}$ of all resources under a static sharing.\footnote{For simplicity of presentation, we consider queues with the same weights, which can be easily extended to queues with different weights.}

\textbf{Strategyproofness (SPF)} ensures that queues cannot benefit by lying about their resource demands. 
This provides incentive compatibility, as a queue cannot improve its allocation by lying. 

\textbf{Pareto efficiency (PE)} is about the optimal utilization of the system. 
A resource allocation is Pareto efficient if it is impossible to increase the allocation/utility of a queue without hurting at least another queue. %This property is important as it leads to maximizing system utilization subject to satisfying the other properties.



\subsection{Analysis of Existing Policies}
\label{sec:property-analysis}


\begin{table}[!t]
	\small
	\centering
	\begin{tabular}{ |c||c|c|c||c| } 
		\hline
		Property  & SP  & DRF  & M-BVT  & BPF \\   [0.5ex] 
		\hline\hline
		Burst Guarantee (BG)    &  \checkmark* &$\times$ & \checkmark*  &\checkmark \\ 
		Long-Term Fairness (LF) & $\times$ & \checkmark &\checkmark &  \checkmark \\
		%Sharing Incentive (SI)  & $\times$ &\checkmark &\checkmark & \checkmark \\ 
		Strategyproofness (SPF)  & $\times$ & \checkmark& $\times$ & \checkmark\\
		Pareto Efficiency (PE)  & \checkmark & \checkmark &\checkmark & \checkmark \\
		\hline \hline 
		Single Resource Fairness & $\times$  & \checkmark & \checkmark & \checkmark \\
		Bottleneck Fairness & $\times$  & \checkmark & \checkmark & \checkmark \\
		Population Monotonicity & \checkmark  & \checkmark & \checkmark & \checkmark \\
		\hline
	\end{tabular}
	\caption{Properties of existing policies and {\name}. $\checkmark^*$ means that the property holds when there is only one \burstq.}
	%\todo{Define the asterisk for BG-SP combination here.}
	\label{tab-properties}
	\vspace{-0.1in}
\end{table}


%\todo{rewrite the following section}

\textbf{Strict Priority (SP): }
SP is employed to provide performance guarantee for {\burstq}s. As the name suggests, an SP scheduler always prioritize {\burstq}s. Therefore, when there is only one \burstq, SP provides the best possible performance guarantee. 
However, when there are more than one {\burstq}s, it is impossible to give all of them the highest priority. Meanwhile, {\batchq}s may not receive enough resources, which violates long-term fairness. As the {\burstq}s may request more resources than what they actually need, strategyproofness is not enforced, and therefore the system may waste some resources -- i.e., it is not Pareto efficient. 

%A danger with this type of service is that a queue can potentially starve out lower-priority traffic classes. 
%
%Obviously strict priority has burst guarantee for LQs and it is pareto-efficient. But for other properties it does not hold. For example, sharing is bad for those queues with lower-priority as they can be delayed for an infinite long time. And it is not a strategy-proofness as it can require all resources by claiming a very high priority level.   

\textbf{DRF:}  
DRF is an extension of max-min fairness to the multi-resource environment, where the dominant share is used to map the resource allocation (as a vector) to a scalar value. 
It provides instantaneous fairness, strategyproofness, and Pareto efficiency. %In particular, the strategyproofness property is straightforward in the single-resource environment but needs careful treatments in the multi-resource environment. 
However, because DRF is an instantaneous allocation policy without any memory, it cannot prioritize jobs with more urgent deadlines. In particular, no burst guarantee is provided. %Meanwhile, DRF is memoryless, it does not consider the time dimension, which is an important bridge between fairness and burstness. 
Even assigning queues different weights in DRF is homogeneous over time and cannot provide the burst guarantee needed. In addition, there is no admission control. Therefore, as the number of queues increases, no queue's performance can be guaranteed.

\textbf{M-BVT: }
BVT~~\cite{bvt} was designed as a scheduler for a mix of real-time and best-effort tasks. The idea is that for real-time tasks, BVT allows them to borrow some virtual time (and therefore resources) from the future and be prioritized for a period without increasing their long-term shares. 

To make it comparable, we extend it to M-BVT for a multi-resource environment.
Under the M-BVT policy, \burstq-$i$ is assigned a virtual time warp parameter $W_i$, which represents the urgency of the queue. Upon an arrival of its burst at $A_i$, an effective virtual time $E_i = A_i-W_i$ is calculated. This is used as the priority (smaller $E_i$ means higher priority) for scheduling. When \burstq-$i$ has the only smallest $E_i$, it may use the whole system's resources and its $E_i$ increases at the rate of its progress calculated by DRF. Eventually, its $E_i$ is no longer the only smallest. Then resources are shared in a DRF-fashion among queues with the smallest virtual times.


%until other queues catch up with \burstq-$i$ or there is some change to the system, e.g., a completion or an arrival of the {\burstq}s. 

The M-BVT policy has some good properties. For instance, the DRF component ensures long-term fairness, and the BVT component strives for performance. Pareto efficiency follows from the work conservation of the policy. 

However, it does not provide general burst guarantees as any new arriving queue (with larger virtual time warp parameter) may occupy the resources of existing {\burstq}s or share resources with them, thus hurting their completion time. In addition, it is not strategyproof because queues can lie about their needs in order to get a larger virtual time warp. 


%DRFQ tries to generalize fair queuing of packets from different flows, where packet consume multiple resources. It is an expansion from DRF and broaden the concept of fairness from space to time dimension with the idea of balancing virtual time, which has some similarity as our contribution. 

%Although DRFQ is used in packet environment, with a bit modification we can also apply it to datacenter scenario. And it have all properties mentioned above except burst guarantee. To allow special treatment for latency-sensitive queues, we need an extra virtual time warp parameter $w$ in BVT paper. The function of $w$ is similar to priority level, but it does not change the long-term share of different queues. However, it's does not have burst guarantee as any new arriving queue with larger $w$ may occupy your resources or share resource with you, thus hurting the completion time. And it's not a strategy-proof as there is no way to prevent user from lying.




\textbf{Other policies} such as the CEEI~\cite{moulin2014cooperative} provide fewer desired properties. 
%\zhenhua{I commented the details of PDRF and CEEI for now}

%\textbf{Proportional Dominant Share Fairness (PDRF):}
%Proportional fairness is a compromise-based scheduling algorithm. It assign different weight to different queues while tries to maximize the total output. We can also apply dominant share here as the following: For TQ$i$, we assign her unit weight and for LQ$j$, we assign her weight $\frac{1}{t}$ where $t$ is the percentage of her ON period divided by her total periodic length.
%
%Then we can directly use DRF to maxmin the weighted dominant share of all queues. Obviously, PDRF is pareto-efficient and sharing incentive. But similar to CEEI, it is not a strategy-proof and does not provide any burst guarantee for LQs. 
%
%
%\textbf{Competitive Equilibrium from Equal Incomes (CEEI):}
%We can also apply the method to fairly divide resources is Competitive Equilibrium from Equal Incomes (CEEI) from microeconomic theory. 
%
%The CEEI allocation is given by the Nash bargaining solution. The Nash bargaining solution picks the feasible allocation that maximizes $\prod_i u_i(a_i)$ where $u_i(a_i)$ is the utility that user $i$ gets from her allocation $a_i$. To support latency-sensitive queues, we can modify the utility function of these LQ queues. For example, we can set the utility for TQ$i$ simply her dominant share $s_i$, the utility for LQ$j$ is $s_i^{t}$, where $t$ is the percentage of her ON period divided by her total periodic length. 
%It's easy to see that CEEI is sharing incentive and pareto-efficient. However, CEEI is not a strategy-proof as she can get more burst speed by claiming a shorter ON period; Also, CEEI does not have burst guarantee for LQ jobs as here no fixed amount of resources is guaranteed for LQ queues.    

\subsection{Summary of the Tradeoffs}

As listed in Table~\ref{tab-properties}, no prior policy can simultaneously provide all the desired properties of fairness/isolation for {\batchq}s while providing burst guarantees for all the {\burstq}s with strategyproofness. In particular, if strict priority is provided to an \burstq without any restriction for its best performance (e.g., SP), there is no isolation protection for {\batchq}s' performance. On the other hand, if the strictly instantaneous fairness is enforced (e.g., DRF), there is no room to prioritize short-term bursts. While the idea in M-BVT is reasonable, it is not strategyproof and cannot provide burst guarantee.

%The key observation that motivates this work is that one cannot simultaneously provide all the desired properties of fairness/isolation for {\batchq}s while providing burst guarantees for all the {\burstq}s (Table~\ref{tab-properties}). 
%In particular, SP provides the best performance (the lowest completion time) to {\burstq}s without any isolation protection for {\batchq}s' performance. 
%DRF provides ideal instantaneous fairness without burst guarantees for {\burstq}s.
%BVT -- even with its multi-resource extension with DRF -- is not strategyproof.
% It is impossible to obtain the best performance and instantaneous fairness at the same time because instantaneous fairness cannot accommodate short-term burst of {\burstq}s.
% Therefore, long-term fairness is enforced instead to give room for the short-term burst requirement for {\burstq}s.
% On the other hand, the burst size of {\burstq}s cannot be too large compared to its long-term fair share.
% For instance, consider a link with capacity 100 MBps and shared by one {\burstq} and one {\batchq}. The {\burstq} has ON status every 10 seconds. If its demand during the ON status is no larger than 500MB (0.5*100MBps*10s), we can prioritize this {\burstq}.
% However, if its demand exceeds 500MB, we cannot give it high priority without hurting {\batchq}'s long-term fair share.

The key question of the paper is, therefore, how to allocate system resources in a near-optimal way; meaning, satisfying all the critical properties in Table~\ref{tab-properties}.

%So the approach should focus on how to dynamically adjust allocations according to the demand and performance metrics.
%
%\textcolor{red}{Key question: does \name solve the problem? If so, what additional challenges should we tackle?}
%
%
%{\burstq}s: high value, time sensitive, hard to accommodate with guarantee
%
%Even with the same utilization level, the system with more {\burstq}s is likely to generate higher profits. 
%However, this makes the resource management more challenging. 
%
%Figure~\ref{fig:mov_example} shows the example for only one {\burstq}, where there is no resource contention among {\burstq}s. 
%This makes the scheduling relatively easier: we just need to give the (only) {\burstq} bounded guarantee over time. 
%With more than one {\burstq}s, naive admission control would result in a suboptimal number of admitted {\burstq}s.
%
%Admission control example:
%
%Resource contention example: add another {\burstq} with arrival at 900 to Figure~\ref{fig:mov_example}. 
%Solution can be SRPT, which decreases JCT.



%\subsubsection*{Challenge: Increase {\burstq} Utilization}
%\label{sec:challenge}
%
%Another metric beyond the performance-isolation tradeoff is the system utilization. However, it is more complicated when {\burstq}s and {\batchq}s coexist. 
%Often, {\burstq}s have high value and are time sensitive. Therefore, they are hard to be accommodated with resource guarantees. On the other hand, it is relatively easy to fill the cluster with (possible) low value, batch jobs.
%In other words, system utilization is no longer the sole metric and the composition of it needs to be examined. 
%
%In this spirit, we call the utilization of {\burstq}s with resource guarantees {\burstq} utilization, which is the third performance metric that operators are likely to be interested in.


%Techniques such as packing can help further increase ``good'' utilization. 
%Our current work is based on admission in an FCFS manner.

% We assume {\burstq}s submit their highest possible demand across bursts.
%\subsection{Challenge: uncertainties in an {\burstq}'s demand}
%
%If an {\burstq} knows the distribution, she can run a stochastic optimization to decide the parameters used in her request. \zhenhua{Do we have any data to support this?}
%
%If an {\burstq} has no knowledge or the distribution varies dramatically over time (such as Figure~\ref{fig:mov_example}), she needs to submit multiple requests over time by solving a (challenging) online problem.

% \subsection{Connections to networking}
% \label{sec:networking}
%
% \todo{add something or remove this subsection}


\section{Online Interchangeable-Resource Allocation Problem}

\subsection{Problem Setting}

\desc{Interchangeability}

In the interchangeable resource allocation problem we considered, there are three resources of interests: CPU, GPU and memory (MEM).
Both CPU and GPU can be used for computational needs in an interchangeable way, MEM demand cannot be replaced by CPU or GPU.
This can be easily extended to include additional resources such as disk I/O, network I/O.

There are $N$ users in the systems. A user has jobs arriving over time.
Job $i$ arrives at time $t_i$, the online profiling tool (\S\ref{sec:res_estimation}) can create 2 configurations estimating the demand and processing time of job $i$ as following: $$
Q_i=
\begin{bmatrix} 
c_i & m_i^1 & p_i^1 \\
g_i & m_i^2 & p_i^2 
\end{bmatrix} $$ where $c_i, m_i^1, p_i^1$ represents the demand for CPU, memory and its estimated processing time from its CPU configuration; $g_i, m_i^2, p_i^2$ represents the demand for GPU, memory and its estimated processing time from its GPU configuration.

Let $C_c, C_g, C_m$ be capacities of CPU cores, GPUs and memory.
$A_c, A_g, A_m$ denotes current loads of the cluster on corresponding resources.

\paragraph{Problem statement} \emph{Given job configurations where jobs can interchange between CPUs and GPUs,
what is the best solution to allocate resources to the jobs such that it maximizes the performance and efficiency while maintains fairness?}

\todo{explain why's it is challenging?}

\subsection{Allocation Policy}

\begin{algorithm}[H]
\small
\caption{\name Online \name Scheduler}
\label{alg:greedy}
\begin{algorithmic}[1]
    \State $C,A:$ System capacity and current load
    \State $L:$ fairness score list of all users
    \State $Queue_i$: all waiting jobs of user $i$ at current time
    \State $U$: user set where the fairness scores of all users in $U$ fall in the lowest $\alpha$\% percentage
    \State $W$: all queueing jobs from set $U$.
    \State $S$: sorted list of all processing times from all jobs in $W$ (sorted increasingly)
    \Procedure{OnlineAlloc()}{}
    \While{$A_c< C_c || A_2 < C_2 ~\&\&~ A_3 < C_3$ }
    \State pick current value $p_n^m$ from list $S$		\Comment{Default is the first value in $S$}
    \If{ $m=2 \&\& A_2+ g_n \leq C_2 ~\&\&~ A_3+m_n^2 \leq C_3$ }  \Comment{GPU config}
    \State{\textsc{JobStart}($L,n$)} 
    \Comment{Place $n$ on GPU}
    \State{Update U,W,S}
    \ElsIf { $m=1 \&\& A_1+ c_n \leq C_1 ~\&\&~ A_3+m_n^1 \leq C_3$ }  \Comment{CPU config}
    \State{\textsc{JobStart}($L,n$)}  \Comment{Place $n$ on CPU}
    \State{Update U,W,S}
    \Else
    \State Move the next job;
    \EndIf
    \EndWhile
    \EndProcedure
    \\
    \Function{JobStart($L,job_j$)}{}  \Comment{Update score if job $j$ starts being processed}
    \State Remove job $j$ from list $W$
    \If {job $j$ is from user $i$}
    \If {$p_j^1 <  p_j^2$} 
    \If {job $j$ is placed on CPU}
    $Value(j) = \max(\frac{c_j}{C_1},\frac{m_j^1}{C_3})$
    \Else ~
    $Value(j) = \frac{p_j^1}{p_j^2}\max(\frac{c_j}{C_1},\frac{m_j^1}{C_3})$
    \EndIf
    \Else
    \If {job $j$ is placed on CPU}
    $Value(j) = \frac{p_j^2}{p_j^1} \max(\frac{g_j}{C_2},\frac{m_j^2}{C_3})$
    \Else ~
    $Value(j) = \max(\frac{g_j}{C_2},\frac{m_j^2}{C_3})$
    \EndIf
    \EndIf
    \EndIf
    \State $L(i) = L(i) + Value(j)$
    \EndFunction     
    \\
    \Function{JobFinish($L,job_j$)}{}  \Comment{Update score if job $j$ finishes}
    \If {job $j$ is from user $i$}
    \State $L(i) = L(i) - Value(j)$
    \EndIf
    \EndFunction
\end{algorithmic}
\end{algorithm}

The allocation policy needs to pack the jobs in the queue across candidate users.
The high level of the policy is as summarized in the following two steps:
(1) Pick a subset of users who receive less than others based on fairness scores.
(2) Schedule jobs of the user subset in to optimize their performance.
The pseudo-code of allocation policy is summarized in Algorithm \ref{alg:greedy}.

To maintain the fairness for step (1), we keep watch on the fairness score of each user.
%As sharing interachanable resources in space domain is not efficient, we need find another way to maintain fairness.
We define the fairness score as \textit{transformed dominant share}.
The idea is the following: Every user wishes his jobs to be finished as fast as possible.
However, that is not doable as some jobs have to sacrifice and take the suboptimal configurations for the performance of other jobs.
The system rewards them for taking a suboptimal solution and finish slower.
Here is how we define \textit{transformed dominant share} value $v(i)$ for a single job $i$. $v(i)$ will be calculated only when it is being processed.
From $$
	Q_i=
	\begin{bmatrix}
	c_i & m_i^1 & p_i^1 \\
	g_i & m_i^2 & p_i^2 
	\end{bmatrix} $$ simply compare $p_i^1$ and $p_i^2$ and check the placement of job $i$. \begin{itemize}
		\item If $p_i^1 <  p_i^2$, and job $i$ is on CPU, then $v(i) = \max(\frac{c_i}{C_1},\frac{m_i^1}{C_3})$.
		\item If $p_i^1 <  p_i^2$, and job $i$ is on GPU, then $v(i) =  (\frac{p_i^1}{p_i^2})^\mu \max(\frac{c_i}{C_1},\frac{m_i^1}{C_3})$.
		\item If $p_i^1 \geq  p_i^2$, and job $i$ is on CPU, then $v(i) =  (\frac{p_i^2}{p_i^1})\mu \max(\frac{g_i}{C_2},\frac{m_i^2}{C_3})$.
		\item If $p_i^1 \geq  p_i^2$, and job $i$ is on GPU, then $v(i) = \max(\frac{g_i}{C_2},\frac{m_i^2}{C_3})$.
	\end{itemize} 
where $\mu \geq 0 $ controls the fairness score when the job is not placed on the preferred resource.
The \textit{transformed dominant share} value of a user is the sum of all values from its jobs that are currently being processed.
\textit{Transformed dominant share} will shrink to DRF if there is only a single configuration.

In step (2), obtaining the optimal schedule is actually a nondeterministic polynomial (NP) time problem. 
We solve this step using heuristic.
A job $i$ has 2 completion times $p^1_i$ and $p^2_i$ on CPU and GPU respectively.
A set with $W$ jobs have $2|W|$ processing times.
We sort the processing times so that we select the job with shortest one to schedule first.
This heuristic is similar to the shortest remaining time first (SRPT) algorithm.

\section{Design Details}
\label{sec:impl}

In this section, we describe how we have implemented \name on Apache YARN, how we use standard techniques for demand estimation, and additional details related to our implementation.

\begin{figure}
\centering
\includegraphics[width=1.0\linewidth]{fig/diagram}
\caption{Enabling bounded prioritization with long-term fairness in a  multi-resource cluster. 
\name-related changes are shown in orange. \todo{change the figure ON durations to deadlines}}
\label{fig:system_design}
\end{figure}

\subsection{Enabling \name in Cluster Managers}
Enabling bounded prioritization with long-term fairness requires implementing the \emph{\name scheduler} itself along with an additional \emph{admission control} module in cluster managers, and it takes \emph{additional information on demand characteristics} from the users. 
A key benefit of {\name} is its simplicity of implementation: we have implemented it in YARN using only $600$ lines of code.
In the following, we describe how and where we have made the necessary changes.

\subsubsection*{Primer on Data-Parallel Cluster Scheduling}
Modern cluster managers typically includes three components: \emph{job manager} or application master (AM), \emph{node manager} (NM), and \emph{resource manager} (RM).

One NM runs on each server in the cluster, and it is responsible for managing resource containers on that server. 
A container is a unit of allocation and are used to run specific tasks. 

For each application, a job manager or AM interacts with the RM to request job demands and receive allocation and progress updates. 
It can run on any server in the cluster. 
AM manages and monitors job demands (memory and CPU) and job status (\texttt{PENDING}, \texttt{IN\_PROGRESS}, or \texttt{FINISHED}). 

The RM is the most important part in terms of scheduling. 
It receives requests from AMs and then schedules resources using an operator-selected scheduling policy. 
It asks NM to prepare resource containers for the various tasks of the submitted jobs.

\subsubsection*{\name Implementation}
We made three changes for taking user input, performing admission control, and calculating resource shares -- all in the RM.
We do not modify NM and AM.
Our implementation also requires more input parameters from the users regarding the demand characteristics of their job queues. 
Figure~\ref{fig:system_design} depicts our design.

\paragraph{User Input} Users submit their jobs to their queues. 
In our system, there are 2 queue types, i.e., {\burstq}s and {\batchq}s. 
We do not need additional parameters for {\batchq}s because they are the same as the conventional queues. 
Hence, we assume that {\batchq}s are already available in the system. 
However, the \name scheduler needs additional parameters for LQs; namely, arrival times and demands.

Users submit a request job that contains their parameters of the new \burstq. 
After receiving the parameters in the job, the RM sets up a new \burstq queue for the user.
Users can also ask the cluster administrator to set up the parameters. 

\paragraph{Admission Control} YARN does not support admission control. 
We implement an admission control module to classify {\burstq}s and {\batchq}s into Hard Guarantee, Soft Guarantee, and Elastic classes. 
A new queue is rejected if it cannot meet the safety condition \eqref{eqn:ad-safety}, which invalids the committed performance.
If it is a \batchq, it is added into the Elastic class.
If the new {\burstq} does not satisfy the fairness condition \eqref{eqn:ad-fair}, it is also admitted to the Elastic class.
If the new {\burstq} meets the fairness condition \eqref{eqn:ad-fair}, but fails at the resource condition \eqref{eqn:ad-enough}, it will be put in the Soft Guarantee class.
If the new {\burstq} meets all the three conditions, i.e., safety, fairness, and resource, it will be admitted to the Hard Guarantee class.

\paragraph{\name Scheduler} We implement \name as a new scheduling policy to achieve our joint goals of bounded priority with long-term fairness. 
Upon registering the queues, users submit their jobs to their {\burstq}s or {\batchq}s. 
Thanks to admission control, {\burstq}s and {\batchq}s are classified into Hard Guarantee, Soft Guarantee, and Elastic classes. 
Note that resource sharing policies are implemented across queues in YARN, jobs in the same queue are scheduled in FIFO manner.
Hence, \name only sets the share at the individual queue level.

\name Scheduler periodically set the share levels to all {\burstq}s in Hard Guarantee and Soft Guarantee classes.
These share levels are actually upper-bounds on resource allocation that an {\burstq} can receive from the cluster. 
Based on the real demand of each {\burstq}, \name allocates resources until it meets the share levels\delete{ (whether it is in the ON period or OFF period)}. 

\name Scheduler allocates the resource to the three classes in the following priority order: (1) Hard Guarantee class, (2) Soft Guarantee class, and (3) Elastic class. 
The {\burstq}s in the Hard Guarantee class are allocated first.
Then, the \name continues allocates the resource to the {\burstq}s in Soft Guarantee class.
The queues in the Elastic class are allocated with left-over resources using DRF \cite{drf}.

%To ensure work conservation, unused resources by all queues are combined and redistributed across queues in the Elastic class using DRF \cite{drf}. 

\subsection{Demand Estimation}

\name requires accurate estimates of resource demands and their durations of \burstq jobs by users. 
These estimations can be done by using well-known techniques; \eg, users can use history of prior runs \cite{rope, jockey, tetris} with the assumption that resource requirements for the tasks in the same stage are similar \cite{drf, pacman, paratimer}. 
We do not make any new contributions on demand estimation in this paper. 
%While {\name} performs the best with accurate estimations, we have found that {\name} is robust to small misestimations in practice (\S\ref{sec:performance_large_scale}).
When LQs have bursty arrivals of different sizes, BPF with the $\alpha$-strategy ensures the performance with the average usage remains similar (\S\ref{sec:performance_large_scale}). 
We consider a more thorough study an important future work. 

\subsection{Operational Issues}

\paragraph{Container Reuse}
Container reuse is a well-known technique that is used in some application frameworks, such as Apache Tez. 
The objective of container reuse is to reduce the overheads of allocating and releasing containers. 
The downside is that it causes resource waste if the container to be reused is larger than the real demand of the new task. 
Furthermore, container reuse is not possible if the new task requires more resource than existing containers. 
For our implementation and deployment, we do not enable container reuse because {\name} periodically prefers more free resources for \burstq jobs, causing its drawbacks to outweigh its benefits in many cases.

\paragraph{Preemption}
Preemption is a recently introduced setting in the YARN Fair Scheduler \cite{hadoop-fair-scheduler}, and it is used to kill running containers of one job to create free containers for another. 
By default, preemption is not enabled in YARN. 
For {\name}, using preemption can help in providing guarantees for {\burstq}s. 
However, killing the tasks of running jobs often results in failures and significant delays. 
We do not use preemption in our system throughout this paper.

%\subsection{Possible issues}

%In the same worker node, containers are sequentially launched. Slow allocation of containers may hurt resource guarantee.

% % old text % %
%\begin{algorithm}
%\caption{Scheduler}
%\label{algorithm1}
%\begin{algorithmic}[1]
%\Procedure{periodicSchedule()}{}
%\State $\{\mathbb{A}, \mathbb{B}, \mathbb{U}\}$ = \textsc{updateQueueStatus}($\mathbb{A}$,$\mathbb{B}$,$\mathbb{U}$)
%\If{\textsc{isNewArrival}($\mathbb{Q}$)}
%	\State Update the best effort queues: $\mathbb{U} = \mathbb{U} \cup \mathbb{Q}$
%\EndIf
%\State $\{\mathbb{A}, \mathbb{B}, \mathbb{U}\}$ = \textsc{admit}($\mathbb{U}$)
%\State \textsc{allocate}($\mathbb{A}$,$\mathbb{B}$)
%\State Obtain the available cluster resources $\myvec{L}$
%\State DRF($\mathbb{U}$, $\myvec{L}$)
%\EndProcedure
%\\
%\Function{updateQueueStatus($\mathbb{A}$,$\mathbb{B}$,$\mathbb{U}$)}{}
%\ForAll{queue $A \in \mathbb{A}$}
%	\If{queue $A$ is deactivated}  Update $\mathbb{A} = \mathbb{A} \setminus A$
%	\EndIf
%\EndFor
%\ForAll{queue $B \in \mathbb{B}$}
%	\If{queue $B$ is deactivated} Update $\mathbb{B} = \mathbb{B} \setminus B$
%	\EndIf	
%\EndFor
%\ForAll{queue $U \in \mathbb{U}$}
%	\If{queue $U$ is deactivated} Update $\mathbb{U} = \mathbb{U} \setminus U$
%	\EndIf
%\EndFor
%\State \textbf{return} $\{\mathbb{A} , \mathbb{B},\mathbb{U}  \}$	
%\EndFunction
%\\
%\Function{admit(queues $\mathbb{U}$)}{}
%\ForAll{queue $U \in \mathbb{U}$}
%	\If{queue $U$ is a bursty queue}
%		\If{admission conditions \eqref{eqn:bursty_adm_cond_2} satisfied}
%			\State Update $\mathbb{A} = \mathbb{A} \cup U$
%			\State Update $\mathbb{U} = \mathbb{U} \setminus U$
%		\EndIf
%	\Else
%		\If{admission condition \eqref{eqn:batch_adm_cond} satisfied}
%			\State Update $\mathbb{B} = \mathbb{B} \cup U$
%			\State Update $\mathbb{U} = \mathbb{U} \setminus U$
%		\EndIf
%	\EndIf
%\EndFor
%\State \textbf{return} $\{\mathbb{A} , \mathbb{B},\mathbb{U}  \}$	
%\EndFunction
%\\
%\Function{\diff{allocate}($\mathbb{A}$, $\mathbb{B}$)}{}
%\ForAll{bursty queue $A \in \mathbb{A}$}
%		\State Compute $\myvec{a_i}$ allocated to queue $A$ as \eqref{eqn:alpha_1}, \eqref{eqn:alpha_2a}
%\EndFor
%\State Obtain the available cluster resources $\myvec{L}$
%\State $\myvec{R_b}$ = \textsc{DRF}($\mathbb{B}$,$\myvec{L}$) as Algorithm 1 of \cite{drf}.  
%\EndFunction
%\end{algorithmic}
%\end{algorithm}


\section{Evaluation}

\paragraph{Setup.} We use the real-trace workloads from Google.

\emph{Workload}

\emph{Experimental setup}

\emph{Simulation setup}

\paragraph{Metrics}

\subsection{Experimental Results}

\subsubsection{Performance Guarantee}

\subsubsection{Long-term Fairness}

\input{scripts/simulations}

\subsection{Results}

%\paragraph{Detailed resource allocation.}
%Figure \ref{fig:res_usage} shows the resource usage and fairness scores of 4 users overtime.
%As sample (sampl.) jobs for all users are highly prioritized as it is necessary to estimate the job completion times.
%It takes 6 minutes in total to finish all the sample jobs.
%Athough all users prefer GPU resources, not all of them are allocated to use GPUs.
%User 1 and User 3 take turns to share CPU as they have lower speed up rates on CPUs.
%user 1's jobs are scheduled first because his jobs on CPU are actually shorter than user 3's.
%The jobs of User 2 and User 4 have higher speed-up rates share GPUs and finish quickly.
%Not the resource usage but the fairscore actually decides the turn of users for resource allocation.
%We set $\alpha=0.5$ here so 2 users are chosen each turn.
%
%\begin{figure}[H]
%	\centering
%	\subfloat[\name]{\includegraphics[width=1.0\linewidth]{figs/res_usage_allox}}
%	\caption{[Cluster] 4 users share the resource and we update fairness scores overtime to maintain the fair allocation. Sampling (sampl.) jobs are prioritized and scheduled first. \todo{fair score \& resource usage are not well synchronized as they were logged using different ways. Need to fix the plot for fairness I just saved the fairness score whenever it changes.}}
%	\label{fig:res_usage}
%\end{figure}

We first evaluate \name through real experiments and validate the accuracy of the simulator in Section~\ref{sec:eval-cluster}. Results from the simulator are discussed in Section~\ref{sec:large-sim}.
% First, we shows the improvement in terms of average completion time in \S\ref{sec:large-sim}.
% Then, we present the trade-offs between performance and fairness of \name in \S\ref{sec:tradeoffs}. 


\subsubsection{Experiments over a cluster}
\label{sec:eval-cluster}

%\paragraph{Average completion time.}
Figure \ref{fig:avgCmplt_exp} illustrates the average job completion time
under \name and other baselines through experiments and the simulator we developed. 
%\todo{Xiao \& Tan, please check the following paragraph to see if it accurately describes the insights. We need more details.}
First, Allox reduces the average completion time significantly compared to other baselines. In particular, \DRFFIFO and \DRFSJF do not fully utilize the CPU resources as shown in Figure~\ref{fig:res_util}, therefore incur longer waiting time. 
\ESRP and \DRFExt reduce the waiting time by increasing the CPU utilization (Figure \ref{fig:res_util}).
%\new{
Although \name has similar CPU and GPU utilization compared to \DRFExt and \ESRP, \name outperforms \DRFExt and \ESRP by better job scheduling and configuration selection. This is highlighted by the significant reduction in job processing time. 
%the placement of \name are more efficient than \DRFExt and \ESRP. 	
%}


%\zhenhua{Can we show the CPU/GPU utilization?} \tanle{added.}



% Although \DRFFIFO it does not suffer from the estimation overheads, the jobs are queued up on GPU and results in the worst performance.
% \DRFSJF performs much better than DRF as jobs are scheduled in the shortest job first manner.
% \ESRP is far from the optimal solution.
% Although we computed the average resource transfer rate for \DRFExt by profiling all the jobs, it does not perform well.

Figure \ref{fig:avgCmplt_exp} validates that the average completion times in the simulation and experiments are consistently similar.
This allows us to perform larger-scale evaluations using the simulator.% more extensive results without running time-consuming experiments.

\begin{figure}[t]
	\centering
%{\includegraphics[width=0.7\linewidth]{figs/avgCmplt_exp}}
{\includegraphics[width=1.0\linewidth]{figs/avg_comparison_group}}
	\caption{[Cluster] \name reduces the average job completion time. For each algorithm, the first bar shows results from experiments, and the second bar is from our simulator.} %Specifically, Allox reduces the completion time by 36\% compared to \DRFFIFO, 24\% compared to \DRFSJF, 25\% compared to \ESRP, and 27\% compared to \DRFExt. For each algorithm, the first bar shows results from experiments, and the second bar is from our simulator.} %Wrong estimation in some jobs in DRFS makes running time are longer than DRFF.}
	\label{fig:avgCmplt_exp}
\end{figure}

%\begin{figure*}[h]
%	\centering
%	\subfloat[\DRFFIFO] {\includegraphics[width=0.15\linewidth]{figs/res_util_DRFF} \label{fig:res_util_DRFF}} 
%	\hspace{0.1cm}
%	\subfloat[\DRFSJF] {\includegraphics[width=0.15\linewidth]{figs/res_util_DRFS} \label{fig:res_util_DRFS}}  
%	\hspace{0.1cm}
%	\subfloat[\ESRP] {\includegraphics[width=0.15\linewidth]{figs/res_util_ES} \label{fig:res_util_ES}} 
%	\hspace{0.1cm} 
%	\subfloat[\DRFExt] {\includegraphics[width=0.15\linewidth]{figs/res_util_DRFExt} \label{fig:res_util_DRFExt}}  
%	\hspace{0.1cm}
%	\subfloat[\name] {\includegraphics[width=0.15\linewidth]{figs/res_util_AlloX} \label{fig:res_util_AlloX}}     
%	\caption{[Cluster] Resource Utilization.}
%	\label{fig:res_util}
%\end{figure*}

\begin{figure}[b]
	\centering
	 \includegraphics[width=0.7\linewidth]{figs/res_util}  
	\caption{[Cluster] CPU and CPU utilization.}
	\label{fig:res_util}
\end{figure}

%\begin{figure}[H]
%	\centering
%{\includegraphics[width=0.7\linewidth]{figs/avg_comparison}}
%	\caption{[Cluster] vs [Simulator]}
%	\label{fig:avgCmplt_exp}
%\end{figure}


%\subsection{Simulation results}


%\emph{Dynamic demand.} Whey jobs are heterogeneous, demand is dynamic. We use simulator to evaluate the performance for demand detection and reallocation.
%\todo{present the dynamic demand case: what traces we use or what random distribution we are based on.}

\subsubsection{Simulation results}
\label{sec:large-sim}

%\todo{introduce other parameters in evaluation if any.}

%\paragraph{Average completion time.}
Figure \ref{fig:sim_large_scale} shows our simulation results. With a larger scale and more jobs, \name ($\alpha = 0.1$) consistently outperforms \DRFFIFO, \DRFSJF, \ESRP, and \DRFExt even more, reducing the completion time by 95\%, 84\%, 88\%, and 53\%, respectively.
Impressively, \name is not far (30\%) from \SRPT that only minimizes the average completion time without considering fairness, and with preemption allowed. 
When we focus on the longest 1\% jobs, \name has even larger improvements and beats \SRPT. This is not surprising because \SRPT prioritizes shorter jobs. 

%\begin{figure}[H]
%	\centering
%	{\includegraphics[width=1\linewidth]{figs/avgCmplt_m1}}    
%	\caption{[Simulation] AlloX outperforms others and is not far from the impractical SRPT in large-scale simulations. \todo{update the text accordingly.}}
%	\label{fig:sim_large_scale}
%\end{figure}

\begin{figure}[h]
	\centering
	\subfloat[all jobs]{\includegraphics[width=0.47\linewidth]{figs/avgCmplt}}
	\subfloat[1\% longest jobs]{\includegraphics[width=0.47\linewidth]{figs/avgCmplt_99}}    
	\caption{[Simulation] AlloX outperforms others and is not far from the impractical SRPT in large-scale simulations.}
	\label{fig:sim_large_scale}
\end{figure}



%\paragraph{Performance overtime} 
%Scheduling jobs to minimize the average completion time may cause large jobs suffering from starvation.
%The error bars in Figure \ref{fig:sim_large_scale} show the average completion times of the top 5\% and 1\% of jobs. 
%Since \DRFSJF, \ESRP, \DRFExt, and \SRPT based on shortest job first schedules the long jobs last, the long-jobs can be starved.
%When there are many queued up jobs like \DRFSJF, the starvation is much more significant. \zhenhua{Why many queued up jobs in \DRFSJF?}
%\new{ \DRFSJF has much longer waiting time than \ESRP because it cannot handle the incoming load that results in more queued up jobs over time.
%}
%\todo{The starvation in SRPT is not noticeable.}


%To provide more details regarding the comparison , we show the job arrivals and average JCT over the time horizon in Figure \ref{fig:perf_overtime}.
%If the average completion times are very high at some periods, there are unscheduled jobs hungry for resources. \zhenhua{Why? Can that be from larger job sizes during those periods or longer queues instead of starvation?}
%\zhenhua{The following description is not clear. Do spikes imply starvation?}
%Recall that \DRFFIFO has the worst performance, it does not have large spikes.
%It is because \DRFFIFO serves the jobs in the FIFO manner so the jobs do not suffer from starvation.
%\DRFSJF has significant spikes.
%Basically, it scarifies starvation for performance. 
%Similarly, \ESRP, \DRFExt, and \SRPT also have noticeable spikes due to starvation.
%In contrast, \name 's average completion time of is the smoothest over time.
%Meaning, \name suffers least from starvation.


\begin{figure}[h]
	\centering
	\hspace{-0.3in}{\includegraphics[width=1.1\linewidth]{figs/perf_overtime}}
	\vspace{-1in}
	\caption{[Simulation] \name consistently maintains the best performance over time, especially during the period with high arrival rates. \name also provides consistently small disparity across users' progress.}
	\label{fig:perf_overtime}
\end{figure}

To provide more details regarding the comparison, we show the job arrivals, average job completion time, and the standard deviation of progresses across users over time in Figure \ref{fig:perf_overtime}.
The average completion times of \name and \SRPT are consistently better over time compared to other baselines. Note the completion time is shown on a logarithmic scale.
When the arrival rates are high, the completion time of other baselines are much higher than that of \name. Not surprisingly, \DRFFIFO cannot process the jobs fast enough so queues build up. 
This highlights by effective scheduling and configuration selection, \name processes jobs faster and therefore allows for high arrival rates. %The large spike around the end is because some jobs running on CPUs finally finish, and artificially increases the average job completion time during this period. 

The figure also shows the disparity across users' progress over time. In this case, \SRPT and \DRFExt are much worse than \name, while DRFF, DRFS, and ES are a little better than \name (all users progress at similar, but much slower rates compared to \name). 
While the disparity in \SRPT is intuitive, \DRFExt fails to provide fairness because it ignores the different speedups of users, and instead use some averaged value to allocate resources. 

Overall, \name provides near optimal performance (similar to \SRPT) and maintains good fairness over time. 

\paragraph{Starvation}

In the extreme case, some users starve under schedulers like \SRPT. 
Figure \ref{fig:job_completed} shows the number of completed jobs of the user with longer jobs than others.
As expected, \SRPT performs the worst with most jobs unfinished.
In contrast, \name provides the best progress of this user because it maintains fairness similar to other baselines and is more effective. 


\begin{figure}[h]
	\centering
	\includegraphics[width=0.7\linewidth]{figs/job_completed}  
	\caption{[Simulation] \name provides the best progress for users with longer jobs, and prevents starvation.}
	\label{fig:job_completed}
\end{figure}

\paragraph{Performance and fairness trade-offs}
\label{sec:tradeoffs}

Figure \ref{fig:fairness_alpha} shows the trade-offs between performance (average completion time) and fairness. 
We vary the parameter $\alpha$ from 0.1 to 1 (smaller means fairer) and compare the performance with \ESRP and \SRPT.
%Small $\alpha$ is better in terms of fairness.
This shows with larger $\alpha$, \name approaches \SRPT in performance. The small gap (9\%) between \name and \SRPT when $\alpha=1$ is due to the fact \SRPT is preemptive at no cost while \name does not allow preemption. 
However, the unfairness in terms of standard deviation of users' progress also increases with larger $\alpha$. Normally, a small $\alpha$ around 0.2 is good at providing large performance improvements at little cost of fairness. 
In practice, a system operator can pick the tradeoff according to her situation. 



%When $\alpha$ is large, \name has more queued jobs for scheduling so it performs better.
%Interestingly, it can reach closely to the performance of unrealistic \SRPT (2.2\%).


%We use Jain's fairness index (JFI) to measure the fairness.
%For fair comparison, we use resource usage as input for JFI for \DRFFIFO, \DRFSJF, \ESRP, and \DRFExt.
%Meanwhile, we use fairness cores as the input for \name and \SRPT.
%\SRPT is the best in terms of average completion time but worst in fairness.
%When we change $\alpha$, \name closes the performance gap with \SRPT.
%Meaning, \name can sacrifice fairness to gain performance.


\begin{figure}[h]
	\centering
	\subfloat[Performance]{{\includegraphics[width=0.47\linewidth]{figs/fairness_alpha}}}
	\hspace{0.05in}
    \subfloat[Fairness]{{\includegraphics[width=0.47\linewidth]{figs/fairness_std_alpha}}}
	\caption{[Simulation] Performance and fairness trade-off. A larger $\alpha$ helps \name approach \SRPT at the cost of worse fairness. }
	\label{fig:fairness_alpha}
\end{figure}

\subsection{Sensitivity Analysis}
\label{sec:sensitivity}

\subsubsection{Estimation errors}
\label{sec:estimation_err}

%Estimation errors are unavoidable.
Figure \ref{fig:analysis_err} evaluates the impacts of mis-estimations on the performance, where we vary the standard deviation of errors from 0 to 60\%.
Even though \DRFFIFO does not need the estimation, its performance is too bad so we compare \name with a stronger baseline \ESRP. %we compare the average completion time of \name to  \DRFFIFO 's.
Even with large estimation errors, \name still provides large improvements that are similar to \SRPT. 
%When the errors are small (under 20\%), the improvements are unchanged.
%When the errors are large (up to 43\%), \name is better than \DRFFIFO.
This highlights the value of incorporating (even noisy) estimation from our estimator. 

\begin{figure}[h]
	\centering
	\includegraphics[width=0.9\linewidth]{figs/analysis_err}
	\caption{[Simulation]  \name is robust to the estimation errors.}%(\DRFSJF is better with large errors because they can utilize more CPUs.) \todo{update the text accordingly.}}
	\label{fig:analysis_err}
\end{figure}

\subsubsection{Profiling overhead}
\label{sec:overhead}

Estimations require running profiling jobs with some overhead. Running a longer sample of a job provides better estimations with larger overhead. 
%The scales of estimation overheads may vary in practice.
Figure \ref{fig:analysis_overhead} evaluates the impacts of profiling overheads on performance. Recall in our experiments have profiling overheads at 3\%.
As \DRFFIFO does not require estimation, its performance is unchanged.% we compare the performance of \name with \DRFFIFO.
With large overheads, the performance of all other solutions including \name and \SRPT degrades. 
However, \name provides consistent, significant improvements compared to other baselines. 
%When the overheads are small (under ??\%), the performance improvement are significant.
In practice, for long and iterative machine learning jobs, it is reasonable to use small sampling jobs. 
We can also obtain the runtime estimation from users \cite{tetrisched, choosy}, which may lead to further improvements.

\begin{figure}[h]
	\centering	
	\includegraphics[width=0.9\linewidth]{figs/analysis_overhead_ext}
	\caption{[Simulation] Impacts of profiling overheads.}% must be small to have the small impacts on performance. Fortunately, sampling jobs can be much smaller than the full jobs as in our experiments. \todo{update the text accordingly.}}
	\label{fig:analysis_overhead}
\end{figure}

\subsubsection{Sensitivity to \name job subset size}
\label{sec:N_max}

Figure \ref{fig:analysis_N_max} shows the performance degradation of \name with smaller numbers of jobs used in the scheduling algorithm. There may be up to 200 jobs available for scheduling. As the figure shows, with a larger capacity of the system, more jobs should be included, but picking a small number already realizes most of the improvements. This allows \name to run at a much faster speed in practice. 
% We also have different capacity sizes.
% The maximum size of queued jobs of a user overtime is 97 (for capacity=20).
% The number of queued jobs linearly increases with the capacity.
% \name quickly converges at a small subset size that allows \name to overcome the computational bottleneck.

\begin{figure}[H]
	\centering
	{\includegraphics[width=0.9\linewidth]{figs/analysis_N_max_ext}}
	\caption{[Simulation] \name achieves most improvements by considering a small subset of available jobs. }
	\label{fig:analysis_N_max}
\end{figure}

\paragraph{Performance of the $\alpha$-strategy.}
\begin{figure*}[!ht]
	\centering
	\subfloat[Percentage of arrivals completed by the deadlines.]{\includegraphics[width=0.3\linewidth]{fig/probabilistic_result} \label{fig:probabilistic_result}}
	\hspace{0.2cm}
	\subfloat[Requested demand normalized by that under the vanilla \name. ]{\includegraphics[width=0.3\linewidth]{fig/probabilistic_provision} \label{fig:probabilistic_provision}}	
	\hspace{0.2cm}
	\subfloat[Resource consumption of LQ.]{\includegraphics[width=0.3\linewidth]{fig/prob_avg_res} \label{fig:prob_avg_res}}
	\caption{[Simulation] The proposed $\alpha$-strategy under $\alpha$=95\% is robust against the uncertainties.}
	\label{fig:probabilistic}
	\vspace{-0.3cm}
\end{figure*}

Figure~\ref{fig:probabilistic} depicts the requested demand, performance, and resource usage under the vanilla \name and the one with the $\alpha$-strategy when arrivals have different sizes. In particular, as the variance increases, the vanilla \name can no longer complete $\alpha$ arrivals before the deadline. Actually, even with 10\% standard deviation, the percentage drops below 50\%. On the other side, \name with $\alpha$-strategy always satisfy the $\alpha$ requirement. Even though the reported demand increases, the average resource usage does not change much, e.g., \batchq still receives the same long-term share.



%\todo{Change the 95 percentile to BPF with $\alpha$-strategy. Change mean to vanilla BPF. Make the 95\% line thicker and remove it from the legend. Change reported demand of (a)'s y-axis to demand. Change (b)'s y-axis to \% of arrivals. Switch the order of (a) and (b)}

\section{Related Work}
\label{sec:related}

\paragraph{Bursty Applications in Big Data Clusters}
Big data clusters experience burstiness from a variety of sources, including periodic jobs \cite{jockey, rope, scarlett, omega}, interactive user sessions \cite{splunk-analysis}, as well as streaming applications \cite{spark-streaming, millwheel, trident}. 
Some of them show predictability in terms of inter-arrival times between successive jobs (\eg, Spark Streaming \cite{spark-streaming} runs periodic mini batches in regular intervals), while some others follow different arrival processes (\eg, user interactions with large datasets \cite{splunk-analysis}).
Similarly, resource requirements of the successive jobs can sometimes be predictable, but often it can be difficult to predict due to external load variations (\eg, time-of-day or similar patterns); the latter, without {\name}, can inadvertently hurt batch queues (\S\ref{sec:motivation}). 

\paragraph{Multi-Resource Job Schedulers}
Although early jobs schedulers dealt with a single resource \cite{late, mantri, quincy}, modern cluster resource managers, \eg, Mesos \cite{mesos}, YARN \cite{yarn}, and others \cite{omega, borg, cosmos}, employ multi-resource schedulers \cite{drf, sjf, tetris, carbyne, apollo, mercury, hdrf} to handle multiple resources and optimize diverse objectives. 
These objectives can be fairness (\eg, DRF \cite{drf}), performance (\eg, shortest-job-first (SJF) \cite{sjf}), efficiency (\eg, Tetris \cite{tetris}), or different combinations of the three (\eg, Carbyne \cite{carbyne}).
% 
However, \emph{all} of these focus on instantaneous objectives, with instantaneous fairness being the most common goal. 
To the best of our knowledge, {\name} is the first multi-resource job scheduler with long-term memory.

\paragraph{Handling Burstiness}
Supporting multiple classes of traffic is a classic networking problem that, over the years, have arisen in local area networks \cite{cbq, intserv-hierarchy, hfsc, diffserv-rfc2475}, wide area networks \cite{bwe, b4, swan}, and in datacenter networks \cite{silo, qjump}. 
All of them employ some form of admission control to provide quality-of-service guarantees.
They also consider only a single link (\ie, a single resource). 
In contrast, {\name} considers multi-resource jobs and builds on top this large body of existing literature.

BVT \cite{bvt} was designed to work with both real-time and best-effort tasks. Although it prioritizes the real-time tasks, it cannot guarantee performance and fairness.

\paragraph{Expressing Burstiness Requirements}
{\name} is not the first system that allows users to express their time-varying resource requirements. 
Similar challenges have appeared in traditional networks \cite{hfsc}, network calculus \cite{cruz1, cruz2}, datacenters \cite{silo, pulsar}, and wide-area networks \cite{bwe}.
Akin to them, {\name} requires users to explicitly provide their burst durations and sizes; {\name} tries to enforce those requirements in short and long terms. 
Unlike them, however, {\name} explores how to allow users to express their requirements in a multi-dimensional space, where each dimension corresponds to individual resources. 
One possible way to collapse the multi-dimensional interface to a single dimension is using the notion of \emph{progress} \cite{hug, drf}; however, progress only applies to scenarios when a user's utility can be captured using Leontief preferences. 


\section{Conclusion}
\label{sec:outro}

%We observe that all existing schedulers force the same performance goal on all jobs, which fails to serve both latency-sensitive {\burstq}s and throughput-sensitive {\batchq}s.
%In fact, existing ``fair'' schedulers \cite{drf, drfq, hdrf,jaffe-maxmin} only make instantaneous decisions causing high latency for {\burstq}s.
%On the other hand, just giving high priority to {\burstq}s does not solve the problem because it may starve the \batchq jobs.
To enable the coexist of latency-sensitive {\burstq}s and the {\batchq}s, we proposed \name (Bounded Priority Fairness).
\name provides bounded performance guarantee to {\burstq}s and maintains the long-term fairness for {\batchq}s.
\name classifies the queues into three classes: Hard Guarantee, Soft Guarantee and Elastic.
\name provides the best performance to {\burstq}s in the Hard Guarantee class and the better performance for {\burstq}s in the Soft Guarantee class. The scheduling is executed in a strategyproof manner, which is critical for public clouds. 
The queues in the Elastic class share the left-over resources to maximize the system utilization. 
In the deployments, we show that \name not only outperforms the DRF up to $5.38\times$ for {\burstq} jobs but also protects {\batchq} jobs up to $3.05\times$ compared to Strict Priority. When {\burstq}'s arrivals have different sizes, adding the $\alpha$-strategy can satisfy the deadlines with similar resource utilization.
%The sensitivity analysis shows that our scheduler is robust to estimation errors and non-preemption.

%\nhattan{We need to summarize the properties of BPF here.}

\phantomsection
\label{EndOfPaper}


%\section{\diff{Potential improvement}}

%\subsection{Improve the utilization of \name}

%We observe that \name may result in \emph{low utilization} if the demand \burstq users are highly unbalanced. For example, because \burstq users require more memory than CPU that makes a large amount of memory unallocated like Figure \ref{fig:admission_speedfair_cluster}.

%The idea to allocate resources like HUG \cite{hug} does instead of DRF \cite{drf} for the leftover resources. However, HUG requires the elastic demands and correlated demand vectors. 
%\begin{itemize}
%	\item We can collect the \emph{correlated demand vectors} from each queue. Hence, we can compute HUG for each elastic queue.
%	\item To have something like elastic demands, we need to implement an \emph{inner-queue scheduler} instead of using FIFO. The idea is schedule the queued up jobs in the elastic queues to \emph{utilize their allocated share}. I think this one is not straight forward because the scheduler needs to choose the \emph{best set of jobs to be allocated}.
%\end{itemize}


%\section{\diff{Next steps}}

%\subsection{support subseconds level for burstiness 2DF}

%\subsection{longer tasks}

%\subsection{memory caching}



%
%
\bibliographystyle{abbrv}
\bibliography{bib/refs}

%\appendix
\section*{Appendix: Proofs for \name Properties}
\label{sec:app}

\begin{proof}[Proof of Theorem~\ref{thm:rest}]%\zhenhua{Xiao, could you please add the proofs? The following paragraphs may be useful.}\xiao{added below,I didn't say much about pareto-efficienncy as in our discussion before, I feel this is enough.} 
The safety condition and fairness condition ensure the long-term fairness for all {\batchq}s. They have no fewer resources than any {\burstq} in the long-term, so it is beneficial for them to share the resources. 

For {\burstq}s in $\mathbb{H}$, they have hard resource guarantee and therefore can meet all the deadlines. For {\burstq}s in $\mathbb{S}$, they have resource guarantee whenever possible, and only need to wait after {\burstq}s in $\mathbb{H}$ when there is a conflict. Therefore, their performance is near optimal and much better than if they were under fair allocation policies. Therefore, both of them have incentives to share. % both of them reach long-term fairness and sharing incentive.

For burst guarantee, it is straightforward from our algorithm that we have the hard guarantee for all {\burstq}s in $\mathbb{H}$, and best-effort guarantee for {\burstq}s in $\mathbb{S}$.

The addition of $\mathbb{S}$ allows more {\burstq}s to be admitted with resource guarantee, and therefore increases the {\burstq} utilization. Finally, we fulfill spare resources with {\batchq}s, so system utilization is maximized, reaching Pareto efficiency.
\end{proof}

\begin{proof}[Proof of Theorem~\ref{thm:spf}]

%\zhenhua{Xiao, I rewrote the following proof. Please check if it works for you.}
Let the true parameters of the LQ be $(\myvec{d},t)$, where $\myvec{d}$ is the demand during bursts and $t$ is the maximum allowed processing time specified in its service level agreements. Let the request parameter be $(\myvec{v},t')$. We first argue that $\myvec{v}=\alpha \myvec{d}$ holds with $\alpha$ being a non-negative scalar.

As $\myvec{d}=\langle d_1,d_2,\cdots,d_k \rangle,\myvec{v}=\langle v_1,v_2,\cdots,v_k \rangle$, let $p= \min_i \left\{\frac{v_i}{d_i}\right\}$. Define a new vector $\myvec{z}= \langle z_1,z_2,\cdots,z_k \rangle$ and let $z_i  = pd_i$. Notice here $\myvec{z}$ has the same performance as $\myvec{v}$, while $z_i \leq v_i$ for any $i$. Therefore, $\myvec{z}$ may request no more resources than $\myvec{v}$, which is more likely to be admitted. Hence, it is always better to request $\myvec{z}$, which is proportional to $\myvec{d}$, the true demand.

%Another thing is that, for user to complete her job in time, we have our deadline condition $t' \leq t$, so it is trivial that $\alpha \geq 1$.

Now consider the two parameters $(\myvec{d},t)$ in \name. If a queue requests more bursty demand ($t'\myvec{v} \geq t\myvec{d}$ with at least one inequality is strictly greater than), it is less likely to be admitted with guarantees. If admitted, it satisfied the deadline constraint either way. Therefore, there is no incentive to request more than what it really needs.

On the other hand, if a queue requests less bursty demand ($t'\myvec{v} \leq t\myvec{d}$ with at least one inequality is strictly smaller than), it is more likely to be admitted with guarantees. However, it may be provided fewer resources than it actually needs to satisfy the deadline constraint. Therefore, it has no incentive to request less than what it really needs either.

The last step is that when a queue requests the true bursty demand ($t'\myvec{v} = t\myvec{d}$), it has no incentive to request a higher or lower rate, i.e., $\myvec{v}=\myvec{d}$. This is parallel to the argument for the total bursty demand: if it requests a higher demand, it still satisfies the deadline if admitted, but is more likely to be rejected; if it requests a lower rate with a larger $t'$, it cannot finish within the deadline even admitted. 

%For the safety condition, these parameters are not involved. For fairness condition, we have $t'\myvec{v} \leq t\myvec{d}$, because if user requests more size, there is a higher chance that the queue will be put in $E$ queue, which is bad for the completion time.

%Another situation is that the user reports more demand and shorter burst time, but they have $t'\myvec{v} = t\myvec{d}$. We claim this is not a dominant strategy as from condition 4, the queue is more likely to be put in S. And for S queues, we don't have guarantee that they can be finished with in their requested time. So in general, the user should report truthfully about their burst demand and duration. 
\end{proof}


%\section{Sigcomm comments}

\begin{itemize}
	\item The proposed solution does not really present any difficult technical challenge. 
	\item BPF requires as input the duration of the ON periods for low latency traffic. It is unclear to me whether jobs can know this in advance so that they can submit it to the scheduler. BFP needs accurate estimates of resource demands and their duration to schedule jobs properly. This looks like a very strong assumption. \nhattan{We can argue for this assumption as our solution is robust to estimation errors and strategyproofness. So, users may not need to report these parameters and they can be estimated instead. This must be justified in the approach section.}
	\item BPF addresses scheduling in one specific scenario.  If new applications or computing abstractions come along, BPF might not generalize.
	\item There is a large body of work we could draw on to schedule latency-sensitive applications on datacenters, but the paper does not seem to take it into account. \nhattan{We indeed have this in related work.}
	\item You should compare BFP to  "SP+DRF", e.g., latency-sensitive jobs get high priority and throughput-sensitive jobs get low priority, the job scheduling within the same priority uses DRF. To guard against latency-sensitive jobs that become too large (e.g., exceeding certain pre-defined threshold), they can be converted into throughput-sensitive jobs. To me, this simple two-tier scheduler will likely achieve the same effect as BFP, and is much easier to implement. \nhattan{This is similar to DRF-W, but it won't provide performance guarantee and it is not fair as latency-sensitive jobs may dominate the resources for a long time.}
\end{itemize}

\section{What we can do?}

\begin{itemize}
	\item 1.5 page reduction. We can remove the Algorithm 1 pseudo code due to page limit and it is covered by Figure Figure 2 as well as text (0.5 pages). We can remove paragraphs at the beginning of section (0.25 pages). Shrink the figures (0.25 pages). Need 0.5 page for simulation results in section 5.3 that duplicate the results in implementation.
	\item Currently we have simulation results and implementation that support long-term fairness, burst guarantee but NOT for pareto efficiency and strategyproofness. There is no easy way to show pareto efficiency in simulation. We can have results for strategyproofness when LQ users report more resources than they need. However, is it too straighforward?	
\end{itemize}

%\theendnotes

\end{document}







