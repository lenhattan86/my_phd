\section{\name: A Scheduler With Memory}
\label{sec:approach}

In this section, we first present the problem settings (\S\ref{sec:settings}) and then formally model the problem in Section~\ref{sec:properties}. \name achieves the desired properties by admission control, guaranteed resource provision, and spare resource allocation presented in Section~\ref{sec:solution_approach}. 
Finally, we prove that BPF satisfies all the properties in Table~\ref{tab-properties} (\S\ref{sec:properties-theorem}).

\subsection{Problem Settings}\label{sec:settings}
% system resource
We consider a system with $K$ types of resources. The capacity of resource $k$ is denoted by $C^k$. The system resource capacity is therefore a vector $\myvec{C} = \langle C^1, C^2, ..., C^K \rangle.$
%where $C^k$ is the capacity for resource $k$ and there are $K$ types of resources in total. 
For ease of exposition, we assume $\myvec{C}$ is a constant over time, while our methodology applies directly to the cases with time-varying $\myvec{C}(t)$, e.g., with estimations of $\myvec{C}(t)$ at the beginning and leveraging stochastic optimization \cite{schneider2007stochastic} and online algorithm design \cite{jaillet2012online}.%, which is beyond the scope of this paper. 

% tenants demand
We restrict our attention to {\burstq}s for interactive sessions and streaming applications, and {{\batchq}}s for batch jobs. 

{\burstq}-$i$'s demand comes from a series $N_i$ of bursts, each consisted of a number of jobs. We denote by $T_i(n)$ the arrival time of the $n$-th burst, which must be finished within $t_i(n)$. Therefore, its $n$-th burst needs to be completed by $T_i(n)+t_i(n)$ (i.e., deadline).
Denote the demand of its $n$-th arrival by a vector $\myvec{d_i}(n)=\langle d^1_i(n), d^2_i(n), ..., d^K_i(n)\rangle$, where $d^k_i(n)$ is the demand on resource-$k$.

In practice, inter-arrival time between consecutive bursts $T_i(n+1)-T_i(n)$ can be fixed for some applications such as Spark Streaming \cite{spark-streaming}, or it may vary for interactive user sessions. In general, the duration is quite short, e.g., several minutes. 
Similarly, the demand vector $\myvec{d_i}(n)$ may contain some uncertainties, and we assume that queues have their own estimations. Therefore, our approach has to be strategyproof so that queues report their estimated demand, %Actually, a stochastic optimization problem can be solved by each {\burstq}-$i$ to decide the vector it wants to report. 
%We take a progressive approach: first, we focus on the case with fixed arrival times and known demand vectors for a simpler solution; then, we enhance that solution to handle the uncertainties.
as well as their true deadlines. 

%Recall that the performance metric of the {\burstq}s is the average completion time of jobs during each ON period. Therefore, each {\burstq} requests the resource demand $\myvec{d_i}(n)$ for a duration that equals to the length of its ON period $t_i(n)$. 

To enforce the long-term fairness, the total demand of {\burstq}-$i$'s $n$-th arrival $\myvec{d_i}(n)$ should not exceed its fair share, which can be calculated by a simple fair scheduler -- i.e., $\frac{\myvec{C}(T_i(n+1)-T_i(n))}{N}$, when there are $N$ queues admitted by {\name} -- or a more complicated one such as DRF. We adopt the former in analysis because it provides a more conservative evaluation of the improvements brought by {\name}. 
% , which is a long-term metric that cannot be satisfied by existing schedulers.
%In particular, traditional fair schedulers such as DRF provides a fixed amount of \emph{guaranteed} resource depending only on the number of tenants in the system. 
%DRF takes into consideration the different demand profiles of tenants to improve progress, while focusing on instantaneous fairness. 
% and throughput.

%To enable long-term performance guarantees, {\burstq}-$i$ provides the system operator a requested a service rate $\myvec{\alpha_i}(n)$ for a duration $t_i(n)$ for each of her arrivals. 
%Note that this is stronger than requesting simply the total amount $\myvec{\alpha_i}(n)*t_i(n)$. 
%If the system operator admits {\burstq}-$i$, the requested rate and duration will be granted, so {\burstq}-$i$ can have better performance.

In contrast, {{\batchq}}s' jobs are queued at the beginning with much larger demand than each burst of {\burstq}s. %In contrast to {\burstq}s who wish to finish each burst as quickly as possible in each ON/OFF period lasting for several minutes, {{\batchq}}s only care about the resources received in the long term, e.g., hours.

%and the total demand is $\myvec{b_j} = \langle b^1_j, b^2_j, ..., b^K_j\rangle$.
%We assume this demand vector is known, but our solution can be applied to the estimated demand vectors with minor changes. 
%She requests the normalized resource profile $\frac{\myvec{b_j}}{\max_k\left\{b^k_j\right\}}$ accordingly. 
%In contrast to {\burstq} tenants, {\batchq} tenants only care about long-term fairness and average completion times.

\begin{table}[!t]
\small
\centering
\begin{tabular}{|c|l|} \hline
Notation & Description \\ \hline \hline
%$\mathbb{A}$ & Admitted {\burstq}s \\ \hline
%$\mathbb{B}$ & Admitted {\batchq}s \\ \hline 
$\mathbb{H}$ & Admitted {\burstq}s with hard guarantee \\ \hline
$\mathbb{S}$ & Admitted {\burstq}s with soft guarantee \\ \hline
$\mathbb{E}$ & Admitted {\batchq}s and {\burstq}s with fair share only \\ \hline
\hline\end{tabular}
\caption{Important notations} 
\vspace{-0.3cm}
\label{tbl:notations}
\end{table}

\subsection{Modeling the Problem}
\label{sec:properties}
%\mosharaf{Rename? this subsection because 3.4 is also about properties?} %\nhattan{Optimization Problem Formulation?}
\paragraph{Completion time:}
Let us denote by $R_i(n)$ the (last) completion time of jobs during {\burstq}-$i$'s $n$-th arrival. If {\burstq}-$i$ is admitted with \emph{hard} guarantee, we ensure that a large fraction ($\alpha_i$)\footnote{$\alpha_i$ can be 95\% or 99\% depending on the SLAs} of arrivals are completed before deadlines, i.e., $\sum_{n\in N_i}\mathbf{1}_{\left\{R_i(n)\leq T_i(n)+t_i(n)\right\}} \geq \alpha_i|N_i|$, where $\mathbf{1}_{\left\{\cdot\right\}}$ is the indicator function which equals to 1 if the condition is satisfied and 0 otherwise, $|N_i|$ is the number of arrivals of {\burstq}-$i$. 
If {\burstq}-$i$ is admitted with only \emph{soft/best-effort} guarantee, we maximize the fraction of arrivals completed on time.

\paragraph{Long-term fairness:}
Denote by $\myvec{a_i}(t)$ and $\myvec{e_j}(t)$ the resources allocated for {\burstq}-$i$ and {\batchq}-$j$ at time $t$, respectively.
For a possibly long evaluation interval $[t, t+T]$ during which there is no new admission or exit, the average resource guarantees received are calculated as $\frac{1}{T}\int_t^{t+T} \myvec{a_i}(\tau)\text{d}\tau$ and $\frac{1}{T}\int_t^{t+T} \myvec{e_j}(\tau)\text{d}\tau$.
We require the allocated dominant resource, i.e., the largest amount of resource allocated across all resource types, received by any {\batchq} queue is no smaller than that received by an {\burstq}. Formally, $\forall i\in\mathbb{A}, \forall j\in\mathbb{B}$, where $\mathbb{A}$ and $\mathbb{B}$ is the set of admitted {\burstq}s and {\batchq}s, respectively, 
\begin{small}$\max_{k}\left\{\frac{1}{T}\int_t^{t+T} a_i^k(\tau)\text{d}\tau\right\}\leq\max_{k}\left\{\frac{1}{T}\int_t^{t+T} e_j^k(\tau)\text{d}\tau\right\}$\end{small}, 
where $a_i^k(\tau)$ and $e_j^k(\tau)$ are allocated type-$k$ resources for {\burstq}-$i$ and {\batchq}-$j$ at time $\tau$, respectively.
This condition provides long-term protections for admitted {\batchq}s.

%\subsubsection*{System utilization and \burstq utilization}
%
%At any time $\tau$, denote by $\myvec{f_i}(\tau)$ and $\myvec{g_j}(\tau)$ the resource consumed by {\burstq}-$i$ and {\batchq}-$j$, respectively. 
%Over the evaluation interval $[t,t+T]$, the average system utilization for all resources can be calculated as follows:
%$$\frac{1}{KT}\sum_k\int_t^{t+T}\frac{\sum_{i\in\mathbb{A}}f_i^k(\tau)+\sum_{i\in\mathbb{B}}g_j^k(\tau)}{C^k}\text{d}\tau, $$
%where $f_i^k(\tau)$ and $g_j^k(\tau)$ are allocated type-$k$ resource for {\burstq}-$i$ and {\batchq}-$j$ at time $\tau$, respectively.
%The \burstq utilization is simply the part consumed by {\burstq}s with guarantees $$\frac{1}{KT}\sum_k\int_t^{t+T}\frac{\sum_{i\in\mathbb{H}\cup\mathbb{S}}f_i^k(\tau)}{C^k}\text{d}\tau. $$

\paragraph{The optimization problem:}

We would like to maximize the arrivals completed before the deadlines for admitted {\burstq}s with soft guarantee while meeting the specified fraction of deadlines of admitted {\burstq}s with hard guarantees and keeping the long-term fairness.
%The problem can be expressed as follows:
%\begin{align}
%  \label{eqn:opt}
%  \begin{aligned}
%    & \min & & \frac{1}{|\mathbb{S}|}\sum_{i\in\mathbb{S}}\left(\frac{1}{N_i}\sum_{n=1}^{N_i}R_i(n)\right)\\
%    & \max & & \frac{1}{KT}\sum_k\int_t^{t+T}\frac{\sum_{i\in\mathbb{A}}f_i^k(\tau)+\sum_{i\in\mathbb{B}}g_j^k(\tau)}{C^k}\text{d}\tau\\
%    & \max & & \frac{1}{KT}\sum_k\int_t^{t+T}\frac{\sum_{i\in\mathbb{H}\cup\mathbb{S}}f_i^k(\tau)}{C^k}\text{d}\tau\\
%    & \text{s.t.} & & R_i(n)\leq T_i(n)+t_i(n), \forall n, \forall i\in\mathbb{H}\\
%    & & & \max_{k}\left\{\frac{1}{T}\int_t^{t+T} a_i^k(\tau)\text{d}\tau\right\}\leq\\
%    & & & \max_{k}\left\{\frac{1}{T}\int_t^{t+T} e_j^k(\tau)\text{d}\tau\right\}, \forall i\in\mathbb{A}, \forall j\in\mathbb{B}
%  \end{aligned}
%\end{align}

The decisions to be made are (i) admission control, which decides the set of admitted {\burstq}s ($\mathbb{H}$, $\mathbb{S}$) and the set of admitted {\batchq}s ($\mathbb{E}$); and (ii) resources allocated to admitted queues {\burstq}-$i$ and {\batchq}-$j$ ($\myvec{a_i}(t)$ and $\myvec{e_j}(t)$, respectively) over time. 
%The resource consumptions ($\myvec{f_i}(t)$ and $\myvec{g_j}(t)$) are different from resource allocations ($\myvec{a_i}(t)$ and $\myvec{e_j}(t)$) because some queues cannot use up the allocated resource, and 
If there are some unused/unallocated resources, queues with unsatisfied demand can share them.%, in addition to their allocated resources. 
%\burstq completion time $R_i(n)$ is some non-increasing function of resource consumption $\myvec{f_i}(t)$, i.e., more resources reduce completion time. 

%\subsection{Solution approach - old}
\label{sec:solution_approach_old}

Our solution \name consists of three major components: admission control, guaranteed resource provisioning, and spare resource allocation. 

\subsubsection{High-level structure}

\begin{itemize}
\item On the arrival of \bursty-$i$:
	\begin{enumerate}
	\item \bursty-$i$ reports $\left(T_i(n), t_i(n), \myvec{\alpha_i}(n)\right)$ for each $n$
	\item \name runs admission control procedure
		\begin{itemize}
		\item If admitted, use guaranteed resource provisioning procedure to set the 				resource guarantee $a_i(t)$
		\item If rejected, put \bursty-$i$ into the best-effort queue
		\end{itemize}
	\end{enumerate}
\item On the arrival of \batch-$j$:
	\begin{enumerate}
	\item \batch-$j$ reports $\frac{\myvec{b_j}}{\max_k\left\{b^k_j\right\}}$
	\item \name runs admission control procedure
		\begin{itemize}
		\item If admitted, use guaranteed resource provisioning procedure to set the 				resource guarantee $e_j(t)$
		\item If rejected, put \batch-$j$ into the best-effort queue
		\end{itemize}
	\end{enumerate}
\item Continuously call the spare resource allocation procedure to allocate leftover resources.
\end{itemize}

\subsubsection{Admission control procedure}
\zhenhua{add reject option}
For \bursty-$i$, we admit it if the following two conditions are satisfied, and reject otherwise.

\begin{align}
	\label{eqn:bursty_adm_cond_2}
    \begin{aligned}
	& \alpha^k_i(n) \leq C^k-\sum_{j\in \mathbb{A}}a^k_j(t), \forall n, k, t\in[T_i(n), T_i(n)+t_i(n)]\\
	& \alpha^k_i(n) t^k_i(n) \leq \frac{C^k}{|\mathbb{A}|+|\mathbb{B}|+1}\left(T_i(n+1)-T_i(n)\right), \forall n, k	 
    \end{aligned}    
\end{align}


The first condition checks if there is still enough uncommitted capacity on each resource to guarantee the requested burst rates.
The second condition holds when the requested total bursty service $\alpha^k_i(n) t^k_i(n)$ does not exceed its share between the $n$-th and $(n+1)$-th arrivals, i.e., $\frac{C^k}{|\mathbb{A}|+|\mathbb{B}|+1}\left(T_i(n+1)-T_i(n)\right)$.

In many cases the \batch tenants are in system at $t=0$. 
However, we make \name more general by admitting \batch tenant over time.

On the arrival of \batch-$j$:
admit if this additional \batch tenant does not violate any existing guarantees:
i.e., 
\begin{align}
	\label{eqn:batch_adm_cond}
	\begin{aligned}
	\alpha^k_i(n) t^k_i(n)\leq\frac{C^k}{|\mathbb{A}|+|\mathbb{B}|+1}\left(T_i(n+1)-T_i(n)\right), \forall i, n,k.
	\end{aligned}
\end{align}
Otherwise, put \batch-$j$ into the best-effort queue.

\subsubsection{Guaranteed resource provisioning procedure}
For each admitted \bursty-$i$, there are two stages for each of its arrivals:

Stage 1: $t\in[T_i(n),T_i(n)+t_i(n)]$, 
\begin{align}
	\label{eqn:alpha_1}
	\begin{aligned}
	\myvec{a_i}(t)=\myvec{\alpha_i}(n);
	\end{aligned}
\end{align}

Stage 2: $t\in[T_i(n)+t_i(n), T_i(n+1)], $
\begin{align}
	\label{eqn:alpha_2}
	\begin{aligned}
	\myvec{a_i}(t)=\frac{\frac{C}{N+M}(T_i(n+1)-T_i(n))-\myvec{\alpha_i}(n) t_i(n)}{T_i(n+1)-T_i(n)-t_i(n)}.
	\end{aligned}
\end{align}

\new{In practice, the bursty queues may not fully receive the guaranteed resources. Ideally, the preemption setting can kill the running jobs or tasks to have more available resorces for bursty queues. However, preemption may lead to significant inefficienty because it has to restart the killed jobs or tasks. On the other hand, a non-preemption setting does not revoke the in-use resoures of running jobs that results in lack of resources when the bursty jobs require.  In oder to handle this problem, we propose to compute $a_i$ in stage 2 as.}

Stage 2: $t\in[T_i(n)+t_i(n), T_i(n+1)], $
\begin{align}
	\label{eqn:alpha_2a}
	\begin{aligned}
	\myvec{a_i}(t) = 
	\begin{cases}
	      \myvec{\alpha_i}(n) & \text{if $\myvec{m}_i(t) < \myvec{\alpha_i}(n) t_i(n)$}  \\  
	      \frac{\frac{C}{N+M}(T_i(n+1)-t)-\myvec{m}_i(t)}{T_i(n+1)-T_i(n)-t} & \text{if $\myvec{m}_i(t) \geq \myvec{\alpha_i}(n) t_i(n)$} .
	\end{cases}
	\end{aligned}
\end{align}
\new{where $\myvec{m}_i(t)=\int_{T(n)}^{t} \myvec{r_i}(\tau)\text{d}\tau$ where $r_i(\tau)$ is the resource usage at time $\tau$. Meaning, we ensure that a bursty queue totally receives enough resources $\myvec{\alpha_i}(n) t_i(n)$ and then maintain the long-term fairness among all the queues}. 

\nhattan{Only one of guarantee levels of resources can be reached in the multiple-resource case.}

\subsubsection{Providing best-effort guarantees to LQs in the best effort queue}
Once the demand of admitted LQs is met, the remaining guarantees are first allocated to the LQs in the best effort queue. We need to do this without hurting the long-term fairness, i.e., the guarantees to TQs.
\zhenhua{add more details}


For each admitted \batch-$j$, we allocate the remaining guarantee $C-\sum_{i\in\mathbb{A}\cup \mathbb{B}}\myvec{a_i}(t)$ using DRF, where $\mathbb{B}$ is the set of best effort LQs.

\vspace{-0.1in}	
\subsection{Solution Approach}
\label{sec:solution_approach}

\name consists of the following three major components:

\noindent\textbf{Admission control procedure:} 
\name classifies admitted {\burstq}s and {\batchq}s into three classes: {\burstq}s admitted with hard resource guarantee ($\mathbb{H}$), {\burstq}s admitted with soft resource guarantee ($\mathbb{S}$), and elastic queues that can be either {\burstq}s or {\batchq}s ($\mathbb{E}$).

Before admitting \burstq-$i$, \name checks if admitting it invalidates any resource guarantees committed for {\burstq}s in $\mathbb{H}\cup\mathbb{S}$, i.e., the \emph{safety condition} $\myvec{d_j}(n) \leq \frac{\myvec{C}\left(T_j(n+1)-T_j(n)\right)}{|\mathbb{H}|+|\mathbb{S}|+|\mathbb{E}|+1}, \forall n, \forall j \in \mathbb{H}\cup\mathbb{S}$.
If it is not satisfied, \burstq-$i$ is rejected. Otherwise, it is safe to admit \burstq-$i$. If its own total demand exceeds its long-term fair share (\emph{fairness condition}), \burstq-$i$ is added to $\mathbb{E}$ for best-effort services. Otherwise if there are enough uncommitted resources, \burstq-$i$ is added to $\mathbb{H}$ with hard guarantee. If there are not enough resources left, it is added to $\mathbb{S}$. 
For {\batchq}-$j$, \name simply checks the safety condition. If it is satisfied, {\batchq}-j is added to $\mathbb{E}$. Otherwise {\batchq}-$j$ is rejected.


\noindent\textbf{Guaranteed resource provisioning procedure}: 
For each {\burstq}-$i$ in $\mathbb{H}$, during $[T_i(n),T_i(n)+t_i(n)]$, \name allocates constant resources to fulfill its demand $\myvec{a_i}(t)=\frac{\myvec{d_i}(n)}{t_i(n)}$. %\footnote{In practice, {\burstq}-$i$ may not fully receive the guaranteed resources. For instance, non-preemption settings do not allow the cluster to kill the running jobs or tasks to give more resources to {\burstq}-$i$ immediately at the beginning of its burst. In such cases, more resources may be provided to {\burstq}-$i$ during the OFF period $[T_i(n)+t_i(n),T_i(n+1)]$ until its resource consumption reaches $\myvec{d_i}(n)t_i(n)$.}. 
{\burstq}s in $\mathbb{S}$ shares the uncommitted resource $\myvec{C}-\sum_{j\in \mathbb{H}}\myvec{a_j}(t)$ based on SRPT~\cite{bansal2001analysis} until each {\burstq}-$i$'s consumption reaches $\myvec{d_i}(n)$ or the deadline arrives. 
Remaining resources are allocated to queues in $\mathbb{E}$ using DRF~\cite{drf}. %\nhattan{do we really need this? It is the same as in Spare resource scheduling.}


\noindent\textbf{Spare resource allocation procedure}:
If some allocated resources are not used, they are further shared by {\batchq}s and {\burstq}s with unsatisfied demand. This maximizes system utilization.








\subsection{Properties of \name}
\label{sec:properties-theorem}

%\todo{make this section more rigorous, proofs in the appendix}

First, we argue that \emph{\name ensures long-term fairness, burst guarantee, and Pareto efficiency}. 
%Corresponding proofs can be found in the appendix.

%\zhenhua{Xiao, could you please add the proofs? The following paragraphs may be useful.}\xiao{added below,I didn't say much about pareto-efficienncy as in our discussion before, I feel this is enough.} 
The safety condition and fairness condition ensure the long-term fairness for all {\batchq}s. %They have no fewer resources than any {\burstq} in the long-term, so it is beneficial for them to share the resources. 

For {\burstq}s in $\mathbb{H}$, they have hard resource guarantee and therefore can meet their SLA. For {\burstq}s in $\mathbb{S}$, they have resource guarantee whenever possible, and only need to wait after {\burstq}s in $\mathbb{H}$ when there is a conflict. Therefore, their performance is much better than if they were under fair allocation policies. %Therefore, both of them have incentives to share. % both of them reach long-term fairness and sharing incentive.

%^For burst guarantee, it is straightforward from our algorithm that we have the hard guarantee for all {\burstq}s in $\mathbb{H}$, and best-effort guarantee for {\burstq}s in $\mathbb{S}$.

The addition of $\mathbb{S}$ allows more {\burstq}s to be admitted with resource guarantee, and therefore increases the system resources utilized by {\burstq}s. Finally, we fulfill spare resources with {\batchq}s, so system utilization is maximized, reaching Pareto efficiency.

In addition, we prove that \emph{\name is strategyproof}. 

%\zhenhua{Xiao, I rewrote the following proof. Please check if it works for you.}
Let the true demand and deadline of LQ-$i$ be $(\myvec{d},t)$ for a particular arrival. Let the request parameter be $(\myvec{v},t')$. We first argue that $\myvec{v}=\beta \myvec{d}$ holds with $\beta>0$.

As $\myvec{d}=\langle d^1,d^2,\cdots,d^k \rangle,\myvec{v}=\langle v^1,v^2,\cdots,v^k \rangle$, let $p= \min_k \left\{\frac{v^i}{d^i}\right\}$. Define a new vector $\myvec{z}= \langle z^1,z^2,\cdots,z^k \rangle$ and let $z^i  = pd^i$. Notice here $\myvec{z}$ has the same performance as $\myvec{v}$, while $z_i \leq v_i$ for any $i$. Therefore, $\myvec{z}$ may request no more resources than $\myvec{v}$, which is more likely to be admitted. Hence, it is always better to request $\myvec{z}$, which is proportional to $\myvec{d}$, the true demand.

%Another thing is that, for user to complete her job in time, we have our deadline condition $t' \leq t$, so it is trivial that $\alpha \geq 1$.
Regarding the demand $\myvec{d}$, reporting a larger $\myvec{v}$ ($\beta>1$) still satisfies its demand, while has a higher risk being rejected as it requests higher demand, while it does not make sense to report a smaller $\myvec{v}$ ($\beta<1$) as it may receive fewer resources than it actually needs. %On the other side, reporting a tighter deadline still satisfies the deadlines, while has a higher risk being rejected as it requests higher demand. 
%Now consider the two parameters $(\myvec{d},t)$ in \name. If a queue requests more bursty demand ($t'\myvec{v} \geq t\myvec{d}$ with at least one inequality is strictly greater than), it is less likely to be admitted with guarantees. If admitted, it satisfied the deadline constraint either way
Therefore, there is no incentive to for LQ-$i$ to lie about its $\myvec{d}$.
The argument for deadline is similar. Reporting a larger deadline does not make sense as it may receive fewer resources than it actually needs. On the other side, reporting a tighter deadline still satisfies the deadlines, while has a higher risk being rejected as it requests higher demand. 
%Now consider the two parameters $(\myvec{d},t)$ in \name. If a queue requests more bursty demand ($t'\myvec{v} \geq t\myvec{d}$ with at least one inequality is strictly greater than), it is less likely to be admitted with guarantees. If admitted, it satisfied the deadline constraint either way
Therefore, there is no incentive to for LQ-$i$ to lie about its deadline, either.

%On the other hand, if a queue requests less bursty demand ($t'\myvec{v} \leq t\myvec{d}$ with at least one inequality is strictly smaller than), it is more likely to be admitted with guarantees. However, it may be provided fewer resources than it actually needs to satisfy the deadline constraint. Therefore, it has no incentive to request less than what it really needs either.
%
%The last step is that when a queue requests the true bursty demand ($t'\myvec{v} = t\myvec{d}$), it has no incentive to request a higher or lower rate, i.e., $\myvec{v}=\myvec{d}$. This is parallel to the argument for the total bursty demand: if it requests a higher demand, it still satisfies the deadline if admitted, but is more likely to be rejected; if it requests a lower rate with a larger $t'$, it cannot finish within the deadline even admitted. 

%For the safety condition, these parameters are not involved. For fairness condition, we have $t'\myvec{v} \leq t\myvec{d}$, because if user requests more size, there is a higher chance that the queue will be put in $E$ queue, which is bad for the completion time.

%Another situation is that the user reports more demand and shorter burst time, but they have $t'\myvec{v} = t\myvec{d}$. We claim this is not a dominant strategy as from condition 4, the queue is more likely to be put in S. And for S queues, we don't have guarantee that they can be finished with in their requested time. So in general, the user should report truthfully about their burst demand and duration. 

\subsection{Handling uncertainties}
\label{sec:uncertainties}

In practice, arrivals of {\burstq}-$i$ may have different sizes, i.e., $\myvec{d_i}$ is not deterministic but instead has some probability distributions. Now we extend \name to handle this case.

We assume {\burstq}-$i$ knows its distributions, e.g., from historical data. In particular, it knows the cumulative probability distribution on each resource $k$, denoted by $F_{ik}$ if these distributions on multiple resources are independent. 
The requirement regarding $\alpha_i$ can be converted into $\prod_j F^{-1}_{ik}(d_{ik}) \geq \alpha_i$, where $d_{ik}$ is the request demand on resource $k$. This gives $F^{-1}_{ik}(d_{ik}) \geq \alpha_i^{1/K}$. Finally the request on resource-$k$ $d_{ik}=F_{ik}(\alpha_i^{1/K})$. We call this $\alpha$-strategy.

When distributions on multiple resources are correlated, we only have the general form $F^{-1}_i (\myvec{d_i}) \geq \alpha_i$, where $F_{i}$ is the joint distribution on all resources. We have the following properties in this case.\footnote{The proof is omitted due to space limit.}
When the distributions are pairwise positively correlated, $\alpha$-strategy over-provisions resources. If they are pairwise negatively correlated, $\alpha$-strategy under-provisions resources. Numerical approaches can be applied to adjust the $\alpha$-strategy accordingly. 

Taking the correlations on multiple resources into consideration is important. In particular, when these distributions are perfectly correlated, $d_{ik}=F_{ik}(\alpha_i^{1/K})$ can be reduced to $d_{ik}=F_{ik}(\alpha_i)$. When the standard deviation is large (e.g., 40\% of the mean for Normal distribution, this can reduce the demand by 10\% with three resources). This increases the chance of {\burstq}-$i$ being admitted.




%Assume CPU and GPU profile has a joint normal distribution with correlation coefficient $\rho$. If we use $\sqrt[n]{\alpha_i}$ as an estimator for $\alpha_i$ percentile of the bivariate normal distribution. Then it's an unbiased estimator if and only if $\rho=0$. When $\rho>0$ it over-estimate the percentile and as $\rho$ approaches to 1, the actual percentile approaches to $\sqrt[n]{\alpha_i}$. When $\rho<0$, it under-estimate the percentile.

%
%$F^{-1}_{ij}(d_{ij})$
%
%
%The goal of the model is firstly, to show that error has an impact on resource allocation; secondly multi-resource is different than single resource.
%
%For now we assume that $T$ is a constant for all users, while resource demand for queue $i$ is $ \vec{d}=(d_{i1},d_{i2},\cdots,d_{in})$ is a set of random variables satisfying i.i.d case \begin{itemize}
%	\item $d_{ij}$ has a cumulative distribution $F_{ij}$ and density function $f_{ij}$;
%	\item $d_{ij}$ and $d_{ik}$ are independent for $j \neq k$.
%\end{itemize}
%
%Or dependent case, i.e they have a joint distribution $f_{i}$.
%
%When a queue arrives, it submits the deadline/duration $t$ and the demand random variable $ \vec{d}$ . The system will calculate the resource required $\vec{a}_i$ for the user to finish $\alpha_i$\% jobs before deadline on average.
%
%
%
%For the required resource to finish $\alpha_i$\% job before the deadline, we have the following constraints for i.i.d case: \begin{equation}
%\prod_j F^{-1}_{ij}(a_{ij}) \geq \alpha
%\end{equation}
%Or \begin{equation} F^{-1}_i (\vec{a}_i) \geq \alpha \end{equation} for the dependent case. We assume the users have normal distribution, so that their joint distribution has a correlation coefficient $\rho$, and their marginal distribution follows normal distribution, which means  $f_{ij}$ is also normal .
%
%
%
%The admission control procedure is (if we only consider LQs) simple. If we can find a solution $\vec{a}_i$ to (1) or (2) and it satisfies other conditions from BPF paper (safety condition, fairness condition etc). Then the user is admitted.
%
%Note: A simple feasible solution to calculate (1) is to let $F^{-1}_{ij}(a_{ij})= \sqrt[n]{\alpha_i}.$ For (2), when the correlation is high, we can simply use every resource $j$'s $\alpha_i$ percentile to approximate the solution (that is we use $\alpha$ instead of $\sqrt[n]{\alpha}$).
%
%when correlation approaches 0, it shrinks to independent case (1).
%\textbf{Missing a proof for $\rho$ to show the figure in next page}.
%\textbf{Missing a approximation function as a better approach compared to $\sqrt[n]{\alpha_i}$ will do before today}.\\
%
%\textcolor{red}{new proof below}
%
%\textbf{Proof 1} Assume CPU and GPU profile has a joint normal distribution with correlation coefficient $\rho$. If we use $\sqrt[n]{\alpha_i}$ as an estimator for $\alpha_i$ percentile of the bivariate normal distribution. Then it's an unbiased estimator if and only if $\rho=0$. When $\rho>0$ it over-estimate the percentile and as $\rho$ approaches to 1, the actual percentile approaches to $\sqrt[n]{\alpha_i}$. When $\rho<0$, it under-estimate the percentile.
%
%\begin{proof}
%	For simplicity we assume that $(X,Y)$ they have a joint standard normal distribution $f$ with correlation coefficient $\rho$. Marginal distribution follows $X \sim N(0,1), Y \sim N(0,1)$ The confidence is $\alpha$.
%	
%	We have for $Y=y$, \begin{equation}
%	\label{eq:conditional}
%		X|y \sim N(\mu_X+\rho\frac{\sigma_X}{\sigma_Y}(y-\mu_Y),\sigma_X^2(1-\rho^2)),
%	\end{equation}
%	If they are independent, they $\rho=0$, then $P(X,Y)=P(X)P(Y)=\sqrt{\alpha}\sqrt{\alpha}=\alpha$, so it's an unbiased estimator. If $\rho \neq 0$,
%
% we wish to evaluate \begin{equation}
%	\label{eq:base}
%	\int_{-\infty}^{\sqrt[2]{\alpha}} \int_{-\infty}^{\sqrt[2]{\alpha}} f(x,y) \,dxdy 
%\end{equation}
%
%From \ref{eq:conditional}, \ref{eq:base} equals 
%\begin{equation}
%	\label{eq:main}
%	\begin{split}
%	Q=\int_{-\infty}^{\sqrt{\alpha}} \int_{-\infty}^{\sqrt{\alpha}} f(x,y) \,dxdy &= \int_{-\infty}^{\sqrt{\alpha}} \int_{-\infty}^{\sqrt{\alpha}} f(x|y)f(y) \,dxdy \\
%	&=	\int_{-\infty}^{\sqrt{\alpha}} \int_{-\infty}^{\sqrt{\alpha}} \phi(\frac{\sqrt{\alpha}-\rho y}{\sqrt{1-\rho^2}})\phi(y) dydy \\
%	&= 	\int_{-\infty}^{\sqrt{\alpha}} \Phi(\frac{\sqrt{\alpha}-\rho y}{\sqrt{1-\rho^2}})\phi(y) dy \\
%	&= \int_{-\sqrt{\alpha}}^{\infty} \Phi(\frac{\sqrt{\alpha}+\rho y}{\sqrt{1-\rho^2}})\phi(y) dy 
%	\end{split}
%\end{equation} 
%	Where $\Phi(\dot)$ is the CDF of a standard normal distribution, $\phi(\dot)$ is the PDF of a standard normal distribution.
%
%When $\alpha$ approaches $\infty$, the integral has exact solution. We have $\int \Phi(\frac{x-\mu}{\sigma})\phi(x) dx = \Phi(\frac{-\mu}{\sqrt{1+\sigma^2}})$ for $\mu$ and $\sigma$ as parameter. "Two theorems for inferences about the normal distribution with applications in acceptance sampling" from B.E.Ellison in 1964.
%
%By the results from B.E.Ellison, when $\alpha$ approaches $\infty$, \ref{eq:main} approaches \begin{equation*}
%	\Phi(\frac{\sqrt{\alpha}/\rho}{\sqrt{1+(\frac{\sqrt{1-\rho^2}}{\rho}})^2}) = \Phi({\sqrt{\alpha}})
%\end{equation*}
% 	Which is the $\sqrt{\alpha}$ percentile. Also, when $\rho$ approaches $1$, $\Phi(\cdot)=1$, $Q \rightarrow \sqrt{\alpha}$ as limit.
% 	
% For the rest cases, take derivative of $\rho$ over integral, \ref{eq:main}  \begin{equation}
% 	\frac{\partial Q}{\partial \rho} = \frac{\exp(\frac{-\alpha^2}{1+\rho})}{2\pi\sqrt{1-\rho^2}}>0
% \end{equation}
%And $Q(0) = \alpha$.	So for fixed $\alpha$, when $\rho>0$, $Q(\rho)>\alpha$, when $\rho<0,Q(\rho)<\alpha$.
%	
%\end{proof}
%
%Approximation:
%
%The $\sqrt{\alpha}$ estimator is good in general, however, when $ \rho \to 1$, the relative error can be $\frac{\sqrt{\alpha}}{\alpha}$
%
%\begin{table}[]
%	\centering
%	\caption{Relative Error changes wrt to $\alpha$}
%	\label{my-label}
%	\begin{tabular}{@{}lllll@{}}
%		\toprule
%		$\alpha$       & 0.8  & 0.9 & 0.95 & 0.99 \\ \midrule
%		Relative Error \% & 11.8 & 5.4 & 2.6  & 0.5  \\ \bottomrule
%	\end{tabular}
%\end{table}
%
