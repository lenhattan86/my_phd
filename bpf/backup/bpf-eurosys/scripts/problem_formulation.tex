\section{Problem formulation (by Tan)}

A arbitrary job requires one or many containers, each container is specified a set of resources (CPU, RAM). Here, we can allocate the number of containers to the job. We consider 3 application types: Streaming, Interactive, and Batch as in Table \ref{tbl:application_type}

\begin{table}[!htp]
	\centering
	\begin{tabular}{@{}lll@{}}
		\toprule\addlinespace
		%\multicolumn{3}{l}{Application type} \\
		\textbf{Application type} & & Input \\ \addlinespace
		\cmidrule{1-3}\addlinespace
		\textbf{Streaming}      &  & $\rho$, execution interval (second)\\
		($N_s$ periodic jobs)	&  & $\delta_s$ maximum delay for the first job\\
				&                       & $\alpha_s$ resources per container \\ \addlinespace 

		\textbf{Interactive}    &          & $\tau$, maximum completion time\\
		(1 single job) 			&  						& $\alpha_i$, resource per container \\ \addlinespace
		
		\textbf{Batch}  &          & $T$, max expected completion time\\
		(1 single job)	& 		   &  $\alpha_b$, resource per container\\ \addlinespace
		\bottomrule
	\end{tabular}
	\label{tbl:application_type}
\end{table}

A streaming application is guaranteed to have a small delay ($<\delta_s$) for the first job, while the processing time for the other jobs must be less than the execution time $\rho$. In the meantime, an interactive application prefer to complete the job as soon as possible ($<\tau$). The batch application having a long processing time must predict the expected completion time $T$. 

\subsection{Running example}

We run 3 applications: streaming at $t=0$, interactive at $t=0$, batch at $t=0$. Assume that their containers require the same amount of resources. The cluster capacity is $30$ containers. If each application is allocated with 100\% resource,
$  $\begin{itemize}
	\item A streaming job (among $N$ periodic jobs of the streaming applications) takes 1 sec --> resource needed per job: 30*1 container-secs.
	\item A interactive job takes 1 sec. --> resource needed per job: 30*1 container-secs.
	\item A batch job takes 5 secs. --> resource needed per job: 30*5=150 container-secs.
\end{itemize}
Figure 5, 6 \& 7 illustrate the service curve for each application.

\begin{figure}
\centering
\begin{tikzpicture}[scale=1]
\begin{axis}[
axis lines = left,
xlabel = secs,
ylabel = containters-secs,
xmin=0,xmax=24,
ymin=0,ymax=150,
]
%Below the red parabola is defined
\addplot [
domain=0:3, 
samples=100, 
color=red,
]
{10*x};
\addplot [
domain=3:24, 
samples=100, 
color=red,
]
{30/12*(x-3)+10*3};
\end{axis}
\end{tikzpicture}
\caption{Service curve for the streaming application with $\rho=12,\delta_s=3$. 30 container-secs are allocated in 3 secs for the first job of $N$ streaming jobs meaning the delay for the streaming application is less than 3 secs. In addition, it ensures that the rest $N-1$ jobs can finish in $\rho=12$ secs.}
\end{figure}

\begin{figure}
	\centering
	\begin{tikzpicture}[scale=1]
	\begin{axis}[
	axis lines = left,
	xlabel = secs,
	ylabel = containters-secs,
	xmin=0,xmax=24,
	ymin=0,ymax=150,
	]
	%Below the red parabola is defined
	\addplot [
	domain=0:5, 
	samples=100, 
	color=red,
	]
	{6*x};
	\addplot [
	domain=5:24, 
	samples=100, 
	color=red,
	]
	{6*5};
	\end{axis}
	\end{tikzpicture}
	\caption{Service curve for the interactive application with $\tau=5$. The interactive application is guaranteed to be allocated enough resource to finish in 5 secs. Beyond 5 secs, there is no guarantee that flattens the service curve.}
\end{figure}

\begin{figure}
	\centering
	\begin{tikzpicture}[scale=1]
	\begin{axis}[
	axis lines = left,
	xlabel = secs,
	ylabel = containters-secs,
	xmin=0,xmax=24,
	ymin=0,ymax=150,
	]
	%Below the red parabola is defined
	\addplot [
	domain=0:5, 
	samples=100, 
	color=red,
	]
	{150/150*x};
	\addplot [
	domain=5:24, 
	samples=100, 
	color=red,
	]
	{150/20*(x-5)+150/150*5};
	\end{axis}
	\end{tikzpicture}
	\caption{Service curve for the batch application with $T=24$. This service curve allows the batch job to finish in 24 secs as expected.}
\end{figure}

